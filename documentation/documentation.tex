\documentclass[color,german]{tudbook}
\usepackage[light,math]{iwona}
\renewcommand*{\dinBold}{\fontencoding{T1}\fontseries{b}\selectfont}
\renewcommand*\thesection{\arabic{section}}
\pagestyle{plain}
\usepackage[utf8]{inputenc}
\usepackage[ngerman]{babel}
\usepackage[hidelinks]{hyperref}
\usepackage{tudthesis, german}
\usepackage[T1]{fontenc}
\usepackage{graphicx}
\usepackage{wrapfig}
\usepackage{hyperref}
\usepackage{float}
\begin{document}

\thesis{Dokumentation}
\author{Paula Sch"oley, Robert Morawa, Gilbert R"ohrbein, Alexandra Wei\ss}
\title{InfoVis}
\supervisedby{Prof. Dr.-Ing. Raimund Dachselt}
\supervisedby{Ulrike Kister, M.Sc., Dipl.-Ing. Ricardo Langner}
\submitdate{8.Juli 2013}
\einrichtung{Fakultät Informatik}
\fachrichtung{}
\institut{Institut für Software- und Multimediatechnik}
\professur{Multimedia-Technologie}
\logofilename{img/imllogo}

\maketitle

\tableofcontents 
\newpage


% ~ ~ ~ ~ ~ ~ ~ ~ ~ ~ ~ ~ ~ ~ ~ ~ ~ ~ ~ ~
% Einführung
% - Kanji Schriftsystem
% - Tube Maps
% - - Eigenschaften
% - - Weitere Verwendungszwecke
% W-Fragen
% - Warum Kanji?
% - Warum Tube Maps?
% - Wer soll es verwenden?
% Designprozess
% - Auswahl eines passenden Datensatzes
% - - Webseiten
% - - Farben
% - - Parallele Handlungsstränge
% - Überlegungen zum Ablauf im Programm
% - - Ugly Wool Spider
% - - Central Station
% - Codierung visueller Variablen
% - Shneiderman Mantra
% - - Overview, .....
% Herausfoderungen
% - Lösungsvorschläge
% Algorithmen
% Framworks
% Offene Tickets
%

% ~ ~ ~ ~ ~ ~ ~ ~ ~ ~ ~ ~ ~ ~ ~ ~ ~ ~ ~ ~


\chapter{Einführung}
\section{Kanji Schriftsystem}
Die japanische Schrift besteht aus drei Alphabeten, wobei eines aus Logogrammen - sogenannten Kanji - besteht und die anderen beiden Silbenalphabete sind. Kanji stammen aus der chinesischen Sprache und wurden von den Japanern übernommen und haben sich seitdem sozusagen parallel weiter entwickelt. In einem Satz werden Kanji Zeichen unter Anderem für den Wortstamm und die Silbenalphabete für das "`Auffüllen"' (beispielsweise Negation oder Zeitformen) verwendet. Hiragana und Katakana, also die Silbenalphabete, können aber als Ersatz für ein Kanji verwendet werden, indem sie die Aussprache (die Lesung) des Kanji beschreiben. \\
Kanji bestehen aus einzelnen Elementen, sogenannten Radikalen, die nach gewissen Regeln miteinander kombiniert werden können. Manche der Radikale sind bereits bedeutungstragend, wie beispielsweise das Radikal \emph{Wasserradikal}, welches für Kontext "`Wasser"' steht. Somit ist es beispielsweise in den Wörtern für "`Flut"' oder auch "`Meer"' enthalten. 
%Bitte Radikal noch einfügen

\section{Tube Maps}
Tube Maps oder Transit Maps (deutsch: Liniennetzplan) sind eine Form der Visualisierung, die für Linien öffentlichen Nahverkehrs - also Busse, U- und Straßenbahnen - verwendet wird. Bisher werden Tube Maps nur von Menschen gefertigt, da viele verschiedene Faktoren das Aussehen und die Eigenschaften der Map beeinflusen.  
\subsection{Eigeschaften}
\begin{itemize}
\item Es existieren \emph{Stationen}, die verschiedene Ausprägungen besitzen können (wie "`Endstation"' oder Station, die einen Umstieg zwischen mehreren Linien ermöglicht). 
\item Stationen liegen auf \emph{Linien}, die diese miteinander verbinden. Linien entsprechen den verlegten Gleisen beziehungsweise Routen, die die einzelnen Stationen verbinden. Es gibt die Möglichkeit, verschiedene Arten von Linien visuell zu unterscheiden, um zum Beispiel zwischen Straßenbahn- und Buslinie zu unterscheiden.
% Linien können sich auch teilen?!
\item In \cite{automaticlayoutmetro08} werden als weiterer Bestandteil der Map \emph{Landmarks}, also Orientierungspunkte wie Flughäfen, Bahnhöfe, oder sehr wichtige touristische Orte, genannt. Dabei helfen sie bei der Orientierung, indem geographische Bezüge zwischen Stationen und Landmarks hergestellt werden.
\item \emph{Zonen} stellen in Tube Maps meist Tarifzonen dar, können aber auch weitere Eigenschaften visualisieren, die sich durch Entfernung codieren lassen. 
\end{itemize}
Tube Maps grenzen sich insofern von Node-Link Diagrammen ab, als dass keine Zyklen entstehen können und Linien sich selten teilen. Ersteres bedeutet auf eine Tube Map übertragen, dass keine Schleifen mit mehreren Stationen existieren (abgesehen von nötigen Wendeschleifen), sondern die Stationen in ihrer festen Reihenfolge bedient werden. 
% hmm die ringbahn in berlin?! mist... aber wie grenzen sich die tube maps sonst ab?

\subsection{Weitere Verwendungszwecke}
\label{tm:verwendungszwecke}
Wie in \cite{automaticlayoutmetro08} beschrieben, gibt es auch weitere Anwendungsgebiete für Tube Maps, wie Projektpläne, Karten von Verbindungen zwischen Objekten wie beispielsweise Programmiersprachen, Aufbau von Webseiten, sowie Metabolic Pathways genannt. Eigene Recherche des Teams hat ähnliche Ergebnisse ergeben, wobei nicht immer bei allen Ergebnissen klar war, warum diese Form der Visualsierung gewählt wurde. 

\chapter{W-Fragen}
% bitte besseren namen einfallen lassen!!!
\section{Warum überhaupt Kanji?}
Die japanische Schrift ist streng hierarchisch und modular aufgebaut. Der Datensatz ist begrenzt, da keine neuen Kanji "`erfunden"' werden, und er lässt sich reduzieren auf sogenannte \emph{Jouyou} Kanji. Jouyou sind Zeichen, die besonders häufig im Alltagsleben verwendet werden, und die somit ein Großteil der Japaner beherrscht. Im Vergleich zur Gesamtzahl aller Kanji beschränken sich die Jouyou auf knapp 2000 Zeichen, was eine starke Reduzierung des Datensatzes ermöglicht.
Diese Reduzierung ist nicht nur von praktischer Bedeutung, es ist außerdem für Interessierte, die die Sprache lernen wollen, ein guter Startpunkt. 

\section{Warum Tube Map für Kanji?}
\begin{itemize}
\item klare "`Anfangsstation"' durch Radikale
\item Exploration ("`welche Radikale sollte ich zuerst lernen?"')
\item Bessere Übersicht über Verlauf als bei einer Darstellung in einem Graphen
\end{itemize}

\section{Wer soll es am Ende verwenden?}
Die Anwendung richtet sich primär an solche, die Kanji lernen möchten. Dabei wird davon ausgegangen, dass entweder ein spezielles Kanji gelernt oder nachgeschlagen wird, oder auch ein generelles Interesse an den Zeichen besteht. Soll das Kanji für "`Baum"' gelernt werden, kann man weitere Kanji lernen, die sehr ähnlich dazu sind. 
\cite{kanjilearningjapanese10} beschreibt eine Lernstrategie, die für Kanji eine besonders gute Anwendung findet. Bei dieser Strategie spielt neben visueller Ähnlichkeit auch das sogenannte "`grouping"' (also die Bildung von Verknüpfungen zwischen Vokabeln basierend auf Bedeutung oder Konzepten) eine Rolle. 
Diese Erkenntnisse lassen vermuten, dass Tube Maps sich zum Lernen von Kanji eignen.

\chapter{Designprozess}
\section{Auswahl eines passenden Datensatzes}
In der ersten Phase des Designprozesses hat das Team sich mit mehreren Datensätzen beschäftigt, die mit Hilfe einer Tube Map visualisiert werden können. Dabei wurden die Datensätze mittels Abstimmung auf vier reduziert, woraufhin sich jedes Teammitglied mit einem Thema beschäftigt hat. Anschließend wurden diese Ergebnisse dem Team vorgestellt und durch Abstimmung dann das Thema für die Visualisierung gewählt
\subsubsection{Webseiten}
Wie im Abschnitt \ref{tm:verwendungszwecke} beschrieben lässt sich die Struktur einer Webseite mittels einer Tube Map darstellen.  Diese Darstellung ist vor Allem für große und komplexe Webseiten von Vorteil, da ein Benutzer sich so einfach eine Übersicht über alle vorhanden Inhalte verschaffen kann. Problematisch dabei ist jedoch, die Eigenschaften der Tube Map deutlich hervorzuheben und nicht einen Graphen als Ergebnis zu erhalten. Durch Verlinkungen zwischen verschiedenen Bereichen von Webseiten wird somit schnell eine der wichtigsten Eigenschaften von Tube Maps verletzt, nämlich die Zyklenfreiheit. \\
Die Idee des Teammitgliedes war daher, die häufig besuchten Webseiten und nicht die Struktur einer Webseite darzustellen. Oft besucht ein Benutzer einen Fundus von Seiten im Laufe des Tages. Diese Routine könnte der Linienführung der Linien entsprechen. Des Weiteren könnten Zonen für die grobe Klassifizierung des Inhalts der Seiten eingeführt werden, wie beispielsweise "`Nachrichten"' oder "`Zeitvertreib"'. Dabei könnten Linien auch bestimmten Schlagwörtern, zum Beispiel "`Technik"', zugeordnet werden, um zusammen mit den Zonen Seiten zu beschreiben, die primär über Neuigkeiten im Bereich von Technik und IT berichten. 

\subsubsection{Farben}
\subsubsection{Parallele Handlungsstränge}

\section{Überlegungen zum Ablauf im Programm}
\subsection{Ugly Wool Spider}
alle kanji zeigen

\subsection{Central Station}
central station ansicht, finales design
wie kamen wir drauf? durch erfahrungen und eigene paper

\section{Codierung visueller Variablen}
Gemäß der in der Vorlesung "`Interaktive Informationsvisualisierung"' besprochenen Effektivität visueller Variablen, werden in Bushu Mappu planare Dimensionen (=Position)(Anwendung der Filter und dadurch Abstand zur Central Station), Größe (Central Station), Farbe (Radikallinien) und Form (Rund: Radikal, Quadrat: Kanji) verwendet. 

\section{shneiderman mantra}
\subsubsection{Overview}
\subsubsection{Zoom}
\subsubsection{Filter}
\subsubsection{Details-on-demand}
\subsubsection{Relate}
\subsubsection{History}
\subsubsection{Extract}

\section{Nicht umgesetzte Konzepte}
\subsection{Zonen für Angabe von zB Strichzahl}

\chapter{Herausfoderungen}
Bei normaler Tube Map:
\begin{itemize}
\item Anzahl Stationen relativ überschaubar
\item initial embedding gegeben durch geographische Lage der Stationen
\item Überschneiden von Linien eher selten (nicht in dem Sinne vorhanden bei U-Bahnen)
\end{itemize}

\section{Lösungsvorschlänge}
Im Folgenden werden Radikale als \emph{Endstationen} und Kanji als \emph{Stationen} bezeichnet, um den Vergleich zu einer üblichen Tube Map leichter zu machen. Dementsprechend verbindet eine Linie eine Endstation mit allen zugehörigen Stationen.
\subsection{Lineare Anordnung von Radikalen}
Bei der Linearen Anordnung werden Endstationen auf einer x-Achse (wie beim kartesischen Koordinatensystem) angeordnet und über ihnen alle Stationen. Diese können nach Kriterien wie Strichanzahl, Schuljahr, JPLT Level oder Häufigkeit auf der y-Achse sortiert werden. Weiterhin sollten diese grob in Cluster aufgeteilt werden, um sie nahe an den Endstationen zu halten, aus denen sie bestehen bzw. zu denen sie gehören. Dies bedeutet, dass die Stationen je nach Menge gleicher gemeinsamer Endstationen einem Cluster zuogeordnet werden.

Hierbei sollte zwischen jeder Station ausreichend Platz gelassen werden, um im Zweifelsfall alle 237 Linien zwischen zwei Stationen hindurch zu führen. Ist das Routing zwischen allen Stationen beendet, so können einige dieser Lücken wieder verkleinert werden, vorausgesetzt, dass sich dadurch nicht (noch mehr) Linien kreuzen. 

\paragraph{Vorteile}
???

\paragraph{Nachteile}
\begin{itemize}
\item sehr großer Platzbedarf
\end{itemize}

\subsection{Radiale Anordnung von Radikalen}

\paragraph{Vorteile}
???

\paragraph{Nachteile}
???

\chapter{Algorithmen}

\chapter{Frameworks}
warum web?
warum d3, coffeescript?

\chapter{Offene Tickets}

% Literatur, Quellen
\bibliographystyle{abbrv}
\bibliography{bibliography}
% Sollten wir auch ein Glossar einfügen, nur um sicher zu gehen?
\end{document}
