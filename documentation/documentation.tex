\documentclass[color,german]{tudbook}
\usepackage[light,math]{iwona}
\renewcommand*{\dinBold}{\fontencoding{T1}\fontseries{b}\selectfont}
\renewcommand*\thesection{\arabic{section}}
\pagestyle{plain}
\usepackage[utf8]{inputenc}
\usepackage[ngerman]{babel}
\usepackage[hidelinks]{hyperref}
\usepackage{tudthesis, german}
\usepackage[T1]{fontenc}
\usepackage{graphicx}
\usepackage{wrapfig}
\usepackage{hyperref}
\usepackage{float}
\begin{document}

\thesis{Dokumentation}
\author{Paula Sch"oley, Robert Morawa, Gilbert R"ohrbein, Alexandra Wei\ss}
\title{InfoVis}
\supervisedby{Prof. Dr.-Ing. Raimund Dachselt}
\supervisedby{Ulrike Kister, M.Sc., Dipl.-Ing. Ricardo Langner}
\submitdate{8.Juli 2013}
\einrichtung{Fakultät Informatik}
\fachrichtung{}
\institut{Institut für Software- und Multimediatechnik}
\professur{Multimedia-Technologie}
\logofilename{img/imllogo}

\maketitle


% Gliederung

% Einführung
%  - Kanji Schriftsystem
%  - Tube Maps
%  - - Eigenschaften
%  - - Weitere Verwendungszwecke
% 
% W-Fragen
%  - Warum Kanji?
%  - Warum Tube Maps?
%  - Wer soll es verwenden?
% 
% Designprozess
%  - Auswahl eines passenden Datensatzes
%  - - Webseiten
%  - - Farben
%  - - Parallele Handlungsstränge
%  - Überlegungen zum Ablauf im Programm
%  - - Ugly Wool Spider
%  - - Central Station
%  - Codierung visueller Variablen
%  - Shneiderman Mantra
%  - - Overview, .....
% 
% Herausfoderungen
%  - Lösungsvorschläge
% 
% Algorithmen
% 
% Framworks
% 
% Offene Tickets



\tableofcontents 
\newpage

\section{Einführung}
\subsection{Kanji Schriftsystem}
\subsection{Tube Maps}

\subsubsection{Eigeschaften}
\subsubsection{Weitere Verwendungszwecke}


\section{W-Fragen}
\paragraph{Warum überhaupt Kanji?}
Die japanische Schrift ist streng hierarchisch und modular aufgebaut. Der Datensatz ist begrenzt, da keine neuen Kanji "`erfunden"' werden, und er lässt sich reduzieren auf sogenannte \emph{Jouyou} Kanji. Jouyou sind Zeichen, die besonders häufig im Alltagsleben verwendet werden, und die somit ein Großteil der Japaner beherrscht. Im Vergleich zur Gesamtzahl aller Kanji beschränken sich die Jouyou auf knapp 2000 Zeichen, was eine große Reduzierung des Datensatzes ermöglicht.
Diese Reduzierung ist nicht nur von praktischer Bedeutung, es ist außerdem für Interessierte, die die Sprache lernen wollen, ein guter Startpunkt. 

\paragraph{Warum Tube Map für Kanji?}
\begin{itemize}
\item klare "`Anfangsstation"' durch Radikale
\item Exploration ("`welche Radikale sollte ich zuerst lernen?"')
\item Bessere Übersicht über Verlauf als bei einer Darstellung in einem Graphen
\end{itemize}

\section{Designprozess}

\subsection{Auswahl eines passenden Datensatzes}

Erste Diskussionen befassten sich in mit der Auswahl des zu visualisierenden Datensets. In einer ausgedehten Brainstorming-Session wurden erste Ideen gesammelt und im Anschluss diskutiert und ausgesiebt. Besonderen Fürspruch erhielten die vier Themen Farben, Websitebesuche, parallele Handlungsstränge und Kanji. Die anderen Vorschläge reichten von der Visualisierung einer Bibliothek mit deren räumlicher Aufteilung als Georeferenz über Email-Diskussionen bis hin zu den Themenschwerpunkten Filmen, unsere Fakultät oder einer Tubemap über Tubemaps. Anschließend wurden die drei vielversprechendsten Themen genauer durcharbeitet, erste Konzepte entwickelt und erneut diskutiert. Folgend werden die drei verworfenen Konzepte vorgestellt.

\subsubsection{Farben}

\subsubsection{Websitebesuche}

\subsubsection{Parallele Handlungsstränge}

Umfassende literarische oder filmische Werke verfügen oft über mehrere Handlungsstränge, die zeitlich parallel verlaufen und einander beeinflussen. Gruppen von Figuren splitten sich häufiger auf und gehen wieder 

\subsection{Überlegungen zum Ablauf im Programm}


\subsection{Wahl der visuellen Variablen}


\subsection{Shneiderman Mantra}

\section{Probleme und Herausfoderungen}
Bei normaler Tube Map:
\begin{itemize}
\item Anzahl Stationen relativ überschaubar
\item initial embedding gegeben durch geographische Lage der Stationen
\item Überschneiden von Linien eher selten (nicht in dem Sinne vorhanden bei U-Bahnen)
\end{itemize}

\subsection{Lösungsvorschlänge}
Im Folgenden werden Radikale als \emph{Endstationen} und Kanji als \emph{Stationen} bezeichnet, um den Vergleich zu einer üblichen Tube Map leichter zu machen. Dementsprechend verbindet eine Linie eine Endstation mit allen zugehörigen Stationen.

\subsubsection{Lineare Anordnung von Radikalen}
Bei der Linearen Anordnung werden Endstationen auf einer x-Achse (wie beim kartesischen Koordinatensystem) angeordnet und über ihnen alle Stationen. Diese können nach Kriterien wie Strichanzahl, Schuljahr, JPLT Level oder Häufigkeit auf der y-Achse sortiert werden. Weiterhin sollten diese grob in Cluster aufgeteilt werden, um sie nahe an den Endstationen zu halten, aus denen sie bestehen bzw. zu denen sie gehören. Dies bedeutet, dass die Stationen je nach Menge gleicher gemeinsamer Endstationen einem Cluster zuogeordnet werden.

Hierbei sollte zwischen jeder Station ausreichend Platz gelassen werden, um im Zweifelsfall alle 237 Linien zwischen zwei Stationen hindurch zu führen. Ist das Routing zwischen allen Stationen beendet, so können einige dieser Lücken wieder verkleinert werden, vorausgesetzt, dass sich dadurch nicht (noch mehr) Linien kreuzen. 

\paragraph{Vorteile}
???

\paragraph{Nachteile}
\begin{itemize}
\item sehr großer Platzbedarf
\end{itemize}

\subsubsection{Radiale Anordnung von Radikalen}

\paragraph{Vorteile}
???

\paragraph{Nachteile}
???


\section{Algorithmen}

\section{Offene Tickets}

% Sollten wir auch ein Glossar einfügen, nur um sicher zu gehen?
\end{document}
