\documentclass[color,german]{tudbook}
\usepackage[light,math]{iwona}
\renewcommand*{\dinBold}{\fontencoding{T1}\fontseries{b}\selectfont}
\renewcommand*\thesection{\arabic{section}}
\pagestyle{plain}
\usepackage[utf8]{inputenc}
\usepackage[ngerman]{babel}
\usepackage[hidelinks]{hyperref}
\usepackage{tudthesis, german}
\usepackage[T1]{fontenc}
\usepackage{graphicx}
\usepackage{wrapfig}
\usepackage{hyperref}
\usepackage{float}
\begin{document}

\thesis{Dokumentation}
\author{Paula Sch"oley, Robert Morawa, Gilbert R"ohrbein, Alexandra Wei\ss}
\title{InfoVis}
\supervisedby{Prof. Dr.-Ing. Raimund Dachselt}
\supervisedby{Ulrike Kister, M.Sc., Dipl.-Ing. Ricardo Langner}
\submitdate{8.Juli 2013}
\einrichtung{Fakultät Informatik}
\fachrichtung{}
\institut{Institut für Software- und Multimediatechnik}
\professur{Multimedia-Technologie}
\logofilename{img/imllogo}

\maketitle


% ~ ~ ~ ~ ~ ~ ~ ~ ~ ~ ~ ~ ~ ~ ~ ~ ~ ~ ~ ~
% Einführung
% - Kanji Schriftsystem
% - Tube Maps
% - - Eigenschaften
% - - Weitere Verwendungszwecke
% W-Fragen
% - Warum Kanji?
% - Warum Tube Maps?
% - Wer soll es verwenden?
% Designprozess
% - Auswahl eines passenden Datensatzes
% - - Webseiten
% - - Farben
% - - Parallele Handlungsstränge
% - Überlegungen zum Ablauf im Programm
% - - Ugly Wool Spider
% - - Central Station
% - Codierung visueller Variablen
% - Shneiderman Mantra
% - - Overview, .....
% Herausfoderungen
% - Lösungsvorschläge
% Algorithmen
% Framworks
%

% ~ ~ ~ ~ ~ ~ ~ ~ ~ ~ ~ ~ ~ ~ ~ ~ ~ ~ ~ ~
\section{Einführung}
\subsection{Kanji Schriftsystem}
Die japanische Schrift besteht aus drei Alphabeten, wobei eines aus Logogrammen - sogenannten Kanji - besteht und die anderen beiden Silbenalphabete sind. Kanji stammen aus der chinesischen Sprache und wurden von den Japanern übernommen und haben sich seitdem parallel entwickelt. In einem Satz werden Kanji Zeichen unter Anderem für den Wortstamm und die Silbenalphabete für das "`Auffüllen"' (beispielsweise Negation oder Zeitformen) verwendet. Hiragana und Katakana, also die Silbenalphabete, können aber auch als Ersatz für ein Kanji verwendet werden, indem sie die Aussprache (die Lesung) des Kanji beschreiben. \\
Kanji bestehen aus einzelnen Elementen, sogenannten Radikalen, die nach gewissen Regeln miteinander kombiniert werden können. Wenige der Radikale sind bereits bedeutungstragend, wie beispielsweise das Radikal \emph{氵}, welches für Kontext "`Wasser"' steht. Somit ist es beispielsweise in den Wörtern für "`Flut"'(沔) oder auch "`Meer"'(海) enthalten. 

\subsection{Tube Maps}
Tube Maps oder Transit Maps (deutsch: Liniennetzplan) sind eine Form der Visualisierung, die für Linien öffentlichen Nahverkehrs - also Busse, U- und Straßenbahnen - verwendet wird. Bisher werden Tube Maps nur von Menschen gefertigt, da viele verschiedene Faktoren das Aussehen und die Eigenschaften der Map beeinflusen.  
\subsubsection{Eigeschaften}
\begin{itemize}
\item Es existieren \emph{Stationen}, die verschiedene Ausprägungen besitzen können (wie "`Endstation"' oder Station, die einen Umstieg zwischen mehreren Linien ermöglicht). 
\item Stationen liegen auf \emph{Linien}, die diese miteinander verbinden. Linien entsprechen den verlegten Gleisen beziehungsweise Routen, die die einzelnen Stationen verbinden. Es gibt die Möglichkeit, verschiedene Arten von Linien visuell zu unterscheiden, um zum Beispiel zwischen Straßenbahn- und Buslinie zu unterscheiden. Des Weiteren dürfen die Linien nur horizontal, vertikal oder im 45 Grad Winkel zueinander sein. 
% Linien können sich auch teilen?!
\item In \cite{automaticlayoutmetro08} werden als weiterer Bestandteil der Map \emph{Landmarks}, also Orientierungspunkte wie Flughäfen, Bahnhöfe, oder sehr wichtige touristische Orte, genannt. Dabei helfen sie bei der Orientierung, indem geographische Bezüge zwischen Stationen und Landmarks hergestellt werden.
\item \emph{Zonen} stellen in Tube Maps meist Tarifzonen dar, können aber auch weitere Eigenschaften visualisieren, die sich durch Entfernung codieren lassen. 
\end{itemize}
Tube Maps grenzen sich insofern von Node-Link Diagrammen ab, als dass keine Zyklen entstehen können und Linien sich selten teilen. Ersteres bedeutet auf eine Tube Map übertragen, dass keine Schleifen mit mehreren Stationen existieren (abgesehen von nötigen Wendeschleifen), sondern die Stationen in ihrer festen Reihenfolge bedient werden. 
% hmm die ringbahn in berlin?! mist... aber wie grenzen sich die tube maps sonst ab?

\subsubsection{Weitere Verwendungszwecke}
\label{tm:verwendungszwecke}
Wie in \cite{automaticlayoutmetro08} beschrieben, gibt es auch weitere Anwendungsgebiete für Tube Maps, wie Projektpläne, Karten von Verbindungen zwischen Objekten wie beispielsweise Programmiersprachen, Aufbau von Webseiten, sowie Metabolic Pathways genannt. Eigene Recherche des Teams hat ähnliche Ergebnisse ergeben, wobei nicht immer bei allen Ergebnissen klar war, warum diese Form der Visualsierung gewählt wurde. 

\section{W-Fragen}
% bitte besseren namen einfallen lassen!!!
\subsection{Warum überhaupt Kanji?}
Die japanische Schrift ist streng hierarchisch und modular aufgebaut. Der Datensatz ist begrenzt, da keine neuen Kanji "`erfunden"' werden, und er lässt sich reduzieren auf sogenannte \emph{Jouyou} Kanji. Jouyou sind Zeichen, die besonders häufig im Alltagsleben verwendet werden, und die somit ein Großteil der Japaner beherrscht. Im Vergleich zur Gesamtzahl aller Kanji beschränken sich die Jouyou auf knapp 2000 Zeichen, was eine starke Reduzierung des Datensatzes ermöglicht.
Diese Reduzierung ist nicht nur von praktischer Bedeutung, es ist außerdem für Interessierte, die die Sprache lernen wollen, ein guter Startpunkt. 

\subsection{Warum Tube Map für Kanji?}
\begin{itemize}
\item klare "`Anfangsstation"' durch Radikale
\item Exploration ("`welche Radikale sollte ich zuerst lernen?"')
\item Bessere Übersicht über Verlauf als bei einer Darstellung in einem Graphen
\end{itemize}

\subsection{Wer soll es am Ende verwenden?}
Die Anwendung richtet sich primär an solche, die Kanji lernen möchten. Dabei wird davon ausgegangen, dass entweder ein spezielles Kanji gelernt oder nachgeschlagen wird, oder auch ein generelles Interesse an den Zeichen besteht. Soll das Kanji für "`Baum"' gelernt werden, kann man weitere Kanji lernen, die sehr ähnlich dazu sind. 
\cite{kanjilearningjapanese10} beschreibt eine Lernstrategie, die für Kanji eine besonders gute Anwendung findet. Bei dieser Strategie spielt neben visueller Ähnlichkeit auch das sogenannte "`grouping"' (also die Bildung von Verknüpfungen zwischen Vokabeln basierend auf Bedeutung oder Konzepten) eine Rolle. 
Diese Erkenntnisse lassen vermuten, dass Tube Maps sich zum Lernen von Kanji eignen.


\section{Designprozess}
\subsection{Auswahl eines passenden Datensatzes}
In der ersten Phase des Designprozesses hat das Team sich mit mehreren Datensätzen beschäftigt, die mit Hilfe einer Tube Map visualisiert werden können. Dabei wurden die Datensätze mittels Abstimmung auf vier reduziert, woraufhin sich jedes Teammitglied mit einem Thema beschäftigt hat. Die Ergebnisse wurden im Team vorgestellt und nach Abstimmung und Diskussion das Thema für die Visualisierung gewählt.

\paragraph{Webseiten}
Wie im Abschnitt \ref{tm:verwendungszwecke} beschrieben lässt sich die Struktur einer Webseite mittels einer Tube Map darstellen.  Diese Darstellung ist vor Allem für große und komplexe Webseiten von Vorteil, da ein Benutzer sich so einfach eine Übersicht über alle vorhanden Inhalte verschaffen kann. Problematisch dabei ist jedoch, die Eigenschaften der Tube Map deutlich hervorzuheben und nicht einen Graphen als Ergebnis zu erhalten. Durch Verlinkungen zwischen verschiedenen Bereichen von Webseiten wird somit schnell eine der wichtigsten Eigenschaften von Tube Maps verletzt, nämlich die Zyklenfreiheit.

Die Idee des Teammitgliedes war daher, die häufig besuchten Webseiten und nicht die Struktur einer Webseite darzustellen. Oft besucht ein Benutzer einen Fundus von Seiten im Laufe des Tages. Diese Routine könnte der Linienführung der Linien entsprechen. Des Weiteren könnten Zonen für die grobe Klassifizierung des Inhalts der Seiten eingeführt werden, wie beispielsweise "`Nachrichten"' oder "`Zeitvertreib"'. Dabei könnten Linien auch bestimmten Schlagwörtern, zum Beispiel "`Technik"', zugeordnet werden, um zusammen mit den Zonen Seiten zu beschreiben, die primär über Neuigkeiten im Bereich von Technik und IT berichten. 

\paragraph{Bildeigenschaften}
Ein weiterer Gedanke verfolgte das Ziel, die Eigenschaften von Bildern wie zum Beispiel Gemälden darzustellen. Hierbei würden die Linien Eigenschaften des Gemäldes und die Stationen das Gemälde selbst darstellen. Es wäre eine Möglichkeit, Muster in dem Gebrauch z.B. von Farbe unter Malern oder Fotographen im Zusammenhang mit ihrer Strömung, der Zeit oder des Bildthemas zu finden. Dabei würde die Tubemap sowohl auf X- und Y-Achse in Zonen aufgeteilt. Die X-Achse wäre in diesem Beispielkonzept eine Zeitachse, die Y-Achse weist die im Werk dominierenden Farben aus; sie würden den typischen Zonen einer Tubemap entsprechen.

Bei diesem Layout würden die Achsen je nach Inhalt gestreckt oder gestaucht und die Linien stetig von einer Hauptlinie abzweigen. Eigenschaften von Gemälden wie die Farbzusammensetzung könnten in Tortendiagrammen dargestellt werden, mehrere Gemälde mit gleichen Eigenschaften zu einer geclustert. Die Auswahl an Bildeigenschaften, die man so kodieren könnte, ist groß. Als Interaktionsmöglichkeiten böte sich an, Künstler oder Skalen zu wechseln sowie diverse Filter oder Neuskalierung vorzunehmen. 

\paragraph{Parallele Handlungsstränge}
Ein dritter Ansatz beschäftigte sich mit parallel verlaufenden Handlungssträngen wie sie beispielsweise in der Fernsehserie Game of Thrones und dessen Buchvorlage vorkommen. Es würde Zuschauern oder Fans der Serie eine Möglichkeit geben, die bisherige Handlung zu rekapitulieren und bisherige Aufeinandertreffen von Charakteren darzustellen. Einzelne Hauptcharaktere erhalten hierbei eine Tubemaplinie, Kapitel bilden Stationen und Handlungsorte die Zonen der Tubemap. Diese würde entlang der X-Achse nach von links nach rechts verlaufen, grob an einem Zeitstrahl orientiert oder aber sich radial in alle Richtungen ausbreiten, wobei das die Vergleichbarkeit erschweren würde.

Stationlabels könnten in dieser Tubemap eine Wortgruppe als Zusammenfassung des Kapitels liefern. Man könnte die Stationen bei Point of View-Kapiteln den PoV auch in der Farbe des jeweiligen Charakters festhalten. Icons könnten Ereignisklassen wie Tode oder kämpferische Auseinandersetzungen symbolisieren. Interaktionen könnten hier die Filterung, das Eingrenzen eines Zeitbereiches und ähnliches beinhalten. 


\subsection{Entwicklung des Programmkonzepts}
\subsubsection{Der gesamte Kanjiraum}
Der erste Gedanke war, dass man die komplette Zahl Kanji anzeigen könnte. Aufgrund der hohen Zahlen an Radikalen würde jedoch ein wichtiges Element der Tubemaps entfallen; die Farbe der einzelnen Linien. Würde man jede Radikallinie einfärben, würden sich die Farben kaum voneinander abheben. Eine Idee war, das mittels Hervorhebung zu lösen, sodass man die Radikallinie selbst oder alle Linien eines Kanji per Selektion einfärben kann. Kleine Labels, die bei traditionellen Tubemaps genutzt werden, die Verkehrslinien innerhalb der Karte zu markieren, werden analog dazu genutzt, um die Radikallinien zu bezeichnen. Zur Unterscheidung von Kanji und Radikalen wird die Form der Station verwendet.

Innerhalb der Ansicht navigierte man vor allem mit Zooming und Panning. Eine Minimap, die die Position des Fensters auf dem Canvas angibt, sorgt hierbei für eine grobe Übersicht. Ideen wurden entworfen, die Kanji hier nach den in ihnen enthaltenen Radikalen zu clustern und diese Cluster erst mittels semantischem Zoom aufzulösen, um die überbordende Komplexität zu verringern und die Analyse nach dem Radikalkriterium zu erleichtern. Mit Filtern hat der Nutzer die Möglichkeit, die Menge der Kanji einzuschränken, hierfür eignen sich Attribute wie die unterschiedichen japanischen Lesungen, die Bedeutungen, Strichzahl oder Häufigkeit des Kanji in japanischen Zeitungsartikeln. Eine Suche sollte Kanji anzeigen, die den ausgewählten Kriterien entsprechen, von dieser Suchliste wäre eine Autonavigation zum gesuchten Kanji möglich gewesen. Das Kanji wird für eine kurze Zeit farblich hervorgehoben, während alle Kanji, auf die die Suchkriterien zutreffen, mit dickerem Rand gezeichnet werden.

Mittels Hover hat der Nutzer die Möglichkeit, sich ein gewähltes Detail eines Kanji in Form eines Tooltips oder Stationslabels darzustellen. Da ein einfacher Hover bei einer komplexeren Tubemap rasch zu ungewünschten Details führen kann, wurde es mit einer kurzen Verweilzeit von ca. einer halben Sekunde kombiniert, sodass eine klare Eingabe des Nutzers vorliegen muss. Um weitere Vergleichbarkeit der Kanji herzustellen führten wir eine ausblendbare Detailtabelle ein. Dort wird pro Spalte ein ausgewähltes Kanji angezeigt, in den Zeilen die dazugehörigen Details. Auch hier sollte man automatisch zu den angezeigten Kanji navigiert werden. 


\subsubsection{Die Central Station-Ansicht}
Nach längerem Überlegen merkten wir, dass die bisherige Struktur des Konzeptes nicht zielführend ist oder sein wird. Man erhält nicht einmal einen groben Überblick über die Kanji, da die Struktur der Tubemap zu groß und komplex ist und man beim Hereinzoomen keinerlei Überblick über die restlichen Kanji erhält. Zudem fehlen unter anderem die für eine Tubemap charakteristischen farbigen Linien. Hinzu kam eine Überlegung, wie Kanji selbst gelernt werden. Normalerweise wird ein Kanji zum Lernen vorgegeben; mittels der Tubemap könnte man "`verwandte"' Kanji, die dieselben Radikale teilen wie das zum Lernen ausgewählte Kanji. Auf diese Weise soll man das Kanji besser in seinen Kontext einbetten können.

In diesem Konzept gibt es einen "`Hauptbahnhof"' (Central Station). Dieses Kanji wird in einer vorherigen Übersicht ausgewählt, in der alle Kanji gelistet sind und gefiltert werden können. Diese Central Station unterscheidet sich signifikant; sie ist größer und zeigt alle Details des Kanji in der Station an. Es werden nur Linien zu den Radikalen angezeigt, die in dem Kanji enthalten sind. Da ein Kanji maximal neun Radikale enthält, ist es nun möglich die Linien mit Farben zu unterscheiden. Auch wird ein Großteil der möglichen Überschneidungen auf diese Weise entfernt. Die Radikallinien führen durch alle Kanji, in denen sie enthalten sind.

Ansonsten bleiben viele Ideen des ursprünglichen Konzeptes erhalten. Farbige, kleine Radikale kennzeichnen die zugehörigen Linien. Kanji können in eine Detailtabelle eingetragen, durchsucht und gefiltert werden, ebenso die Autonavigation. Jedoch entfällt auf diese Weise das Hervorheben der Linien. Stationenbezeichner erfahren eine kleine Änderung; in der Central Station existiert nun die Möglichkeit, Detailkategorien auszuwählen, die dort angezeigt werden. Zudem wird eine weitere Interaktionsmöglichkeit mit den Kanji eingeführt. Man kann nun ein Kanji anwählen, um es zu einer neuen Central Station werden zu lassen. Dabei sollen die Radikallinien, die bereits layoutet werden, weil sie in beiden Kanji enthalten sind, nur minimal verändert werden, während die neu hinzugekommenen Radikallinien ins Layout eingefügt werden. Nicht mehr benötigte Radikallinien werden ausgeblendet.

\subsection{Codierung visueller Variablen}
Gemäß der in der Vorlesung "`Interaktive Informationsvisualisierung"' besprochenen Effektivität visueller Variablen, werden in Bushu Mappu planare Dimensionen (=Position)(Anwendung der Filter und dadurch Abstand zur Central Station), Größe (Central Station), Farbe (Radikallinien) und Form (Rund: Radikal, Quadrat: Kanji) verwendet. 

\subsection{shneiderman mantra}
\subsubsection{Overview}
\subsubsection{Zoom}
\subsubsection{Filter}
\subsubsection{Details-on-demand}
\subsubsection{Relate}
\subsubsection{History}
\subsubsection{Extract}

\subsection{Nicht umgesetzte Konzepte}
\subsubsection{Zonen für Angabe von zB Strichzahl}

\section{Herausfoderungen}
Da Haltestellen einen festen Ort haben, der sich nicht "andert, wird dieser als sehr wichtiger Faktor ("`initial embedding"') f"ur das Layouting in "`Automatic Layout of Metro Maps using Multicriteria Optimisation"' von \cite{automaticlayoutmetro08} beschrieben. Dabei kann sich die Position geringf"ugig ver"andern, zum Beispiel werden Entfernungen zwischen Haltestellen auf dem Plan teilweise nicht in dem eigentlich vorhanden Abstand dargestellt, sondern der Abstand wird angeglichen, um ein uniformes Aussehen des Plans zu gew"ahrleisten. Diese Verortung existiert bei Kanji in diesem Ma"se nicht und daher m"ussen Richtungen und Positionen durch einen Algorithmus festgelegt werden. 

"Ahnlich verh"alt es sich mit der Kreuzung von Linien au"serhalb von Haltestellen. Vor allem bei Schienenverkehr findet eine solche Kreuzung selten statt, meist bedienen Linien stattdessen mindestens eine gemeinsame Haltestelle. Da eine physische Begrenzung dieser Art bei dem verwendeten Datensatz nicht vorhanden ist, m"ussen Linien dort aufwendig per Routingalgorithmus gelegt werden, um "Uberschneidungen – soweit m"oglich – zu vermeiden.

Da der Datensatz mit knapp 2000 Kanji deutlich gr"o"ser ist als beispielsweise die Anzahl der Stra"senbahnhaltestellen der Dresdner Verkehrsbetriebe(259 laut \cite{dvbag}), musste vor der Implementierung ein Konzept entwickelt werden, wie die Daten automatisch in Form einer Tube Map korrekt dargestellt werden k"onnen.


\section{Algorithmen}
% über den optimierer

\section{Frameworks}

warum web?
warum d3, coffeescript?



% Literatur, Quellen
\bibliographystyle{abbrv}
\bibliography{bibliography}
% Sollten wir auch ein Glossar einfügen, nur um sicher zu gehen?
\end{document}
