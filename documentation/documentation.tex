\documentclass[color,german]{tudbook}
\usepackage[light,math]{iwona}
\renewcommand*{\dinBold}{\fontencoding{T1}\fontseries{b}\selectfont}
\renewcommand*\thesection{\arabic{section}}
\pagestyle{plain}
\usepackage[utf8]{inputenc}
\usepackage[ngerman]{babel}
\usepackage[hidelinks]{hyperref}
\usepackage{tudthesis, german}
\usepackage[T1]{fontenc}
\usepackage{graphicx}
\usepackage{wrapfig}
\usepackage{hyperref}
\usepackage{float}
\usepackage{CJK}
\begin{document}

\thesis{Dokumentation}
\author{Paula Sch"oley, Robert Morawa, Gilbert R"ohrbein, Alexandra Wei\ss}
\title{InfoVis}
\supervisedby{Prof. Dr.-Ing. Raimund Dachselt}
\supervisedby{Ulrike Kister, M.Sc., Dipl.-Ing. Ricardo Langner}
\submitdate{8.Juli 2013}
\einrichtung{Fakultät Informatik}
\fachrichtung{}
\institut{Institut für Software- und Multimediatechnik}
\professur{Multimedia-Technologie}
\logofilename{img/imllogo}

\maketitle


% ~ ~ ~ ~ ~ ~ ~ ~ ~ ~ ~ ~ ~ ~ ~ ~ ~ ~ ~ ~
% Einführung
% - Kanji Schriftsystem
% - Tube Maps
% - - Eigenschaften
% - - Weitere Verwendungszwecke
% W-Fragen
% - Warum Kanji?
% - Warum Tube Maps?
% - Wer soll es verwenden?
% Designprozess
% - Auswahl eines passenden Datensatzes
% - - Webseiten
% - - Farben
% - - Parallele Handlungsstränge
% - Überlegungen zum Ablauf im Programm
% - - Ugly Wool Spider
% - - Central Station
% - Codierung visueller Variablen
% - Shneiderman Mantra
% - - Overview, .....
% Herausfoderungen
% - Lösungsvorschläge
% Algorithmen
% Framworks
%
% ~ ~ ~ ~ ~ ~ ~ ~ ~ ~ ~ ~ ~ ~ ~ ~ ~ ~ ~ ~

\begin{CJK}{UTF8}{min}
\section{Einführung}
\subsection{Kanji Schriftsystem}
Die japanische Schrift besteht aus drei Alphabeten, wobei eines aus Logogrammen - sogenannten Kanji - besteht und die anderen beiden Silbenalphabete sind. Kanji stammen aus der chinesischen Sprache und wurden von den Japanern "ubernommen und haben sich seitdem parallel entwickelt. In einem Satz werden Kanji unter anderem f"ur den Wortstamm und die Silbenalphabete f"ur grammatikalische Beugung (beispielsweise Negation oder Zeitformen) verwendet. Hiragana und Katakana, also die Silbenalphabete, k"onnen aber auch als Ersatz f"ur ein Kanji verwendet werden, indem sie die Aussprache (die Lesung) des Kanji beschreiben. \\
Kanji bestehen aus einzelnen Elementen, sogenannten Radikalen, die miteinander kombiniert werden k"onnen. Wenige der Radikale sind bereits bedeutungstragend, wie beispielsweise das Radikal \emph{氵}, welches f"ur Kontext "`Wasser"' steht. Somit ist es beispielsweise in den W"ortern f"ur "`Flut"'(沔) oder auch "`Meer"'(海) enthalten. 

\subsection{Tube Maps}
Tube Maps oder Transit Maps (deutsch: Liniennetzplan) sind eine Form der Visualisierung, die f"ur Linien "offentlichen Nahverkehrs - also Busse, U- und Stra"senbahnen - verwendet wird. Bisher werden Tube Maps nur von Menschen gefertigt, da viele verschiedene Faktoren das Aussehen und die Eigenschaften der Tube Map beeinflusen.  
\subsubsection{Eigeschaften}
\begin{itemize}
\item Es existieren \emph{Stationen}, die verschiedene Auspr"agungen, wie "`Endstation"' oder "`Umstiegsm"oglichkeiten"' besitzen. 

\item Stationen liegen auf \emph{Linien}, die diese miteinander gemeinsam haben. Linien entsprechen den verlegten Gleisen beziehungsweise Routen, die die einzelnen Stationen verbinden. Es gibt die M"oglichkeit, verschiedene Arten von Linien visuell zu unterscheiden, um zum Beispiel zwischen Stra\ss enbahn- und Buslinie zu unterscheiden. Des Weiteren d"urfen die Linien nur horizontal, vertikal oder im 45$^{\circ}$ Winkel zueinander sein. 

Linien k"onnen sich ab einer Station aufteilen, um zu zwei alternativen Endstationen zu fahren. 

Hinzu kommen Ringbahnen, bei denen die Endstation der Anfangsstation entspricht.

\item In \cite{automaticlayoutmetro08} werden als weiterer Bestandteil der Map \emph{topographische Metadaten}, also Orientierungspunkte wie Fl"usse, Flugh"afen, Bahnh"ofe, oder sehr wichtige touristische Orte genannt. Dabei helfen sie bei der Orientierung, indem geographische Bez"uge zwischen Stationen und Landmarks hergestellt werden.

\item Die Beschriftung der vorhandenen Stationen und Linien eines Liniennetzes spielt eine wichtige Rolle beim Kennenlernen und Verinnerlichen der vorhandenen Linien. H"aufig wird eine Farbcodierung f"ur die Beschriftung der Linien verwendet, w"ahrend dies bei Stationen nicht der Fall ist und diese direkt mit Text versehen werden. 

\item \emph{Zonen} stellen in Tube Maps meist Tarifzonen dar, k"onnen aber auch weitere Eigenschaften visualisieren, die sich durch Entfernung kodieren lassen. 
\end{itemize}

\subsubsection{Weitere Verwendungszwecke}
\label{tm:verwendungszwecke}
Wie in \cite{automaticlayoutmetro08} beschrieben, gibt es auch weitere Anwendungsgebiete f"ur Tube Maps, wie Projektpl"ane, Karten von thematischen Verbindungen zwischen B"uchern \cite{oreilly}, Aufbau von Webseiten, sowie Metabolic Pathways. Eigene Recherche des Teams hat "ahnliche Ergebnisse ergeben, wobei nicht immer bei allen Visualisierungen klar war, warum eine Visualisierung in Form einer Tube Map gew"ahlt wurde oder sich die Daten besoders dafür eignen.

\section{Motivation}
\subsection{Warum Kanji?}
Die japanische Schrift ist streng hierarchisch und modular aufgebaut. Der Datensatz ist begrenzt, da keine neuen Kanji "`erfunden"' werden, und er l"asst sich reduzieren auf sogenannte \emph{Jouyou} Kanji. Jouyou sind Zeichen, die besonders h"aufig im Alltagsleben verwendet werden, und die somit ein Gro\ss teil der Japaner beherrscht. Im Vergleich zur Gesamtzahl aller Kanji (das Team verwendet eine Datei mit 12000 Kanji) beschr"anken sich die Jouyou auf knapp 2000 Zeichen, was eine starke Reduzierung des Datensatzes erm"oglicht.
Diese Reduzierung ist nicht nur von praktischer Bedeutung, es ist au\ss erdem f"ur Interessierte, die die Sprache lernen wollen, ein guter Startpunkt. 

\subsection{Warum eine Tube Map f"ur Kanji?}
Durch den modularen Aufbau von Kanji k"onnen Radikale als Endstationen interpretiert werden, deren Linien durch mehrere Stationen (also Kanji) verlaufen. Dadurch ergibt sich der Aufbau eines Kanji anhand der Linien, die die Station bedienen. 

Lernende k"onnen somit existierendes Wissen "uber Radikale und Kanji zur Orientierung verwenden, oder aber neue Zeichen kennenlernen, indem sie der Linie eines bekannten Radikals folgen. Anhand der H"aufigkeitsstatistik, die f"ur die Alltags-Kanji vorhanden ist, kann ein Benutzer sich anhand h"aufiger Kanji darin enthaltene Radikale merken und diese verwenden, um weitere Kanji zu lernen. 
\paragraph{Wer soll es am Ende verwenden?}
Die Anwendung richtet sich prim"ar an solche, die Kanji lernen m"ochten. Dabei wird davon ausgegangen, dass entweder ein spezielles Kanji gelernt oder nachgeschlagen wird, oder auch ein generelles Interesse an den Zeichen besteht. Soll beispielsweise das Kanji f"ur "`Baum"' gelernt werden, kann man weitere Kanji lernen, die sehr "ahnlich dazu sind.  
\cite{kanjilearningjapanese10} beschreibt eine Lernstrategie, die f"ur Kanji eine besonders gute Anwendung findet. Bei dieser Strategie spielt neben visueller "Ahnlichkeit auch das sogenannte "`grouping"' (also die Bildung von Verkn"upfungen zwischen Vokabeln basierend auf Bedeutung oder Konzepten) eine Rolle. Diese Erkenntnisse lassen vermuten, dass Tube Maps sich zum Lernen von Kanji eignen.


\section{Designprozess}
\subsection{Auswahl eines passenden Datensatzes}
In der ersten Phase des Designprozesses hat das Team sich mit mehreren Datens"atzen besch"aftigt, die mit Hilfe einer Tube Map visualisiert werden k"onnen. Dabei wurden die Datens"atze mittels Abstimmung auf vier reduziert, woraufhin sich jedes Teammitglied mit einem Thema besch"aftigt hat. Die Ergebnisse wurden im Team vorgestellt und nach Abstimmung und Diskussion das Thema f"ur die Visualisierung gew"ahlt.

\paragraph{Webseiten}
Wie im Abschnitt \ref{tm:verwendungszwecke} beschrieben l"asst sich die Struktur einer Webseite mittels einer Tube Map darstellen.  Diese Darstellung ist vor Allem f"ur gro"se und komplexe Webseiten von Vorteil, da ein Benutzer sich so einfach eine "Ubersicht "uber alle vorhanden Inhalte verschaffen kann. Problematisch dabei ist jedoch, die Eigenschaften der Tube Map deutlich hervorzuheben und nicht einen Graphen als Ergebnis zu erhalten. Durch Verlinkungen zwischen verschiedenen Bereichen von Webseiten wird somit schnell eine der wichtigsten Eigenschaften von Tube Maps verletzt, n"amlich die Zyklenfreiheit.

Die Idee des Teammitgliedes war daher, die h"aufig besuchten Webseiten und nicht die Struktur einer Webseite darzustellen. Oft besucht ein Benutzer einen Fundus von Seiten im Laufe des Tages. Diese Routine k"onnte der Linienf"uhrung der Linien entsprechen. Des Weiteren k"onnten Zonen f"ur die grobe Klassifizierung des Inhalts der Seiten eingef"uhrt werden, wie beispielsweise "`Nachrichten"' oder "`Zeitvertreib"'. Dabei k"onnten Linien auch bestimmten Schlagw"ortern, zum Beispiel "`Technik"', zugeordnet werden, um zusammen mit den Zonen Seiten zu beschreiben, die prim"ar "uber Neuigkeiten im Bereich von Technik und IT berichten. 

\paragraph{Bildeigenschaften}
Ein weiterer Gedanke verfolgte das Ziel, die Eigenschaften von Bildern wie zum Beispiel Gem"alden darzustellen. Hierbei w"urden die Linien Eigenschaften des Gem"aldes und die Stationen das Gem"alde selbst darstellen. Es w"are eine M"oglichkeit, Muster in dem Gebrauch z.B. von Farbe unter Malern oder Fotographen im Zusammenhang mit ihrer Str"omung, der Zeit oder des Bildthemas zu finden. Dabei w"urde die Tubemap sowohl auf X- und Y-Achse in Zonen aufgeteilt. Die X-Achse w"are in diesem Beispielkonzept eine Zeitachse, die Y-Achse weist die im Werk dominierenden Farben aus; sie w"urden den typischen Zonen einer Tubemap entsprechen.

Bei diesem Layout w"urden die Achsen je nach Inhalt gestreckt oder gestaucht und die Linien stetig von einer Hauptlinie abzweigen. Eigenschaften von Gem"alden wie die Farbzusammensetzung k"onnten in Tortendiagrammen dargestellt werden, mehrere Gem"alde mit gleichen Eigenschaften zu einer geclustert. Die Auswahl an Bildeigenschaften, die man so kodieren k"onnte, ist gro"s. Als Interaktionsm"oglichkeiten b"ote sich an, K"unstler oder Skalen zu wechseln sowie diverse Filter oder Neuskalierung vorzunehmen. 

\paragraph{Parallele Handlungsstr"ange}
Ein dritter Ansatz besch"aftigte sich mit parallel verlaufenden Handlungsstr"angen wie sie beispielsweise in der Fernsehserie Game of Thrones und dessen Buchvorlage vorkommen. Es w"urde Zuschauern oder Fans der Serie eine M"oglichkeit geben, die bisherige Handlung zu rekapitulieren und bisherige Aufeinandertreffen von Charakteren darzustellen. Einzelne Hauptcharaktere erhalten hierbei eine Tubemaplinie, Kapitel bilden Stationen und Handlungsorte die Zonen der Tubemap. Diese w"urde entlang der X-Achse nach von links nach rechts verlaufen, grob an einem Zeitstrahl orientiert oder aber sich radial in alle Richtungen ausbreiten, wobei das die Vergleichbarkeit erschweren w"urde.

Stationlabels k"onnten in dieser Tubemap eine Wortgruppe als Zusammenfassung des Kapitels liefern. Man k"onnte die Stationen bei Point of View-Kapiteln den PoV auch in der Farbe des jeweiligen Charakters festhalten. Icons k"onnten Ereignisklassen wie Tode oder k"ampferische Auseinandersetzungen symbolisieren. Interaktionen k"onnten hier die Filterung, das Eingrenzen eines Zeitbereiches und "ahnliches beinhalten. 


\subsection{Entwicklung des Programmkonzepts}
\subsubsection{Der gesamte Kanjiraum}
Der erste Gedanke war, dass man die komplette Zahl Kanji anzeigen k"onnte. Aufgrund der hohen Zahlen an Radikalen w"urde jedoch ein wichtiges Element der Tubemaps entfallen; die Farbe der einzelnen Linien. W"urde man jede Radikallinie einf"arben, w"urden sich die Farben kaum voneinander abheben. Eine Idee war, das mittels Hervorhebung zu l"osen, sodass man die Radikallinie selbst oder alle Linien eines Kanji per Selektion einf"arben kann. Kleine Labels, die bei traditionellen Tubemaps genutzt werden, die Verkehrslinien innerhalb der Karte zu markieren, werden analog dazu genutzt, um die Radikallinien zu bezeichnen. Zur Unterscheidung von Kanji und Radikalen wird die Form der Station verwendet.

Innerhalb der Ansicht navigierte man vor allem mit Zooming und Panning. Eine Minimap, die die Position des Fensters auf dem Canvas angibt, sorgt hierbei f"ur eine grobe "Ubersicht. Ideen wurden entworfen, die Kanji hier nach den in ihnen enthaltenen Radikalen zu clustern und diese Cluster erst mittels semantischem Zoom aufzul"osen, um die "uberbordende Komplexit"at zu verringern und die Analyse nach dem Radikalkriterium zu erleichtern. Mit Filtern hat der Nutzer die M"oglichkeit, die Menge der Kanji einzuschr"anken, hierf"ur eignen sich Attribute wie die unterschiedichen japanischen Lesungen, die Bedeutungen, Strichzahl oder H"aufigkeit des Kanji in japanischen Zeitungsartikeln. Eine Suche sollte Kanji anzeigen, die den ausgew"ahlten Kriterien entsprechen, von dieser Suchliste w"are eine Autonavigation zum gesuchten Kanji m"oglich gewesen. Das Kanji wird f"ur eine kurze Zeit farblich hervorgehoben, w"ahrend alle Kanji, auf die die Suchkriterien zutreffen, mit dickerem Rand gezeichnet werden.

Mittels Hover hat der Nutzer die M"oglichkeit, sich ein gew"ahltes Detail eines Kanji in Form eines Tooltips oder Stationslabels darzustellen. Da ein einfacher Hover bei einer komplexeren Tubemap rasch zu ungew"unschten Details f"uhren kann, wurde es mit einer kurzen Verweilzeit von ca. einer halben Sekunde kombiniert, sodass eine klare Eingabe des Nutzers vorliegen muss. Um weitere Vergleichbarkeit der Kanji herzustellen f"uhrten wir eine ausblendbare Detailtabelle ein. Dort wird pro Spalte ein ausgew"ahltes Kanji angezeigt, in den Zeilen die dazugeh"origen Details. Auch hier sollte man automatisch zu den angezeigten Kanji navigiert werden. 


\subsubsection{Die Central Station-Ansicht}
Nach l"angerem "Uberlegen merkten wir, dass die bisherige Struktur des Konzeptes nicht zielf"uhrend ist oder sein wird. Man erh"alt nicht einmal einen groben "Uberblick "uber die Kanji, da die Struktur der Tubemap zu gro"s und komplex ist und man beim Hereinzoomen keinerlei "Uberblick "uber die restlichen Kanji erh"alt. Zudem fehlen unter anderem die f"ur eine Tubemap charakteristischen farbigen Linien. Hinzu kam eine "Uberlegung, wie Kanji selbst gelernt werden. Normalerweise wird ein Kanji zum Lernen vorgegeben; mittels der Tubemap k"onnte man "`verwandte"' Kanji, die dieselben Radikale teilen wie das zum Lernen ausgew"ahlte Kanji. Auf diese Weise soll man das Kanji besser in seinen Kontext einbetten k"onnen.

In diesem Konzept gibt es einen "`Hauptbahnhof"' (Central Station). Dieses Kanji wird in einer vorherigen "Ubersicht ausgew"ahlt, in der alle Kanji gelistet sind und gefiltert werden k"onnen. Diese Central Station unterscheidet sich signifikant; sie ist gr"o"ser und zeigt alle Details des Kanji in der Station an. Es werden nur Linien zu den Radikalen angezeigt, die in dem Kanji enthalten sind. Da ein Kanji maximal neun Radikale enth"alt, ist es nun m"oglich die Linien mit Farben zu unterscheiden. Auch wird ein Gro"steil der m"oglichen "Uberschneidungen auf diese Weise entfernt. Die Radikallinien f"uhren durch alle Kanji, in denen sie enthalten sind.

Ansonsten bleiben viele Ideen des urspr"unglichen Konzeptes erhalten. Farbige, kleine Radikale kennzeichnen die zugeh"origen Linien. Kanji k"onnen in eine Detailtabelle eingetragen, durchsucht und gefiltert werden, ebenso die Autonavigation. Jedoch entf"allt auf diese Weise das Hervorheben der Linien. Stationenbezeichner erfahren eine kleine "Anderung; in der Central Station existiert nun die M"oglichkeit, Detailkategorien auszuw"ahlen, die dort angezeigt werden. 

Zudem wird eine weitere Interaktionsm"oglichkeit mit den Kanji eingef"uhrt. Man kann nun ein Kanji anw"ahlen, um es zu einer neuen Central Station werden zu lassen. Dabei sollen die Radikallinien, die bereits layoutet werden, weil sie in beiden Kanji enthalten sind, nur minimal ver"andert werden, w"ahrend die neu hinzugekommenen Radikallinien ins Layout eingef"ugt werden. Nicht mehr ben"otigte Radikallinien werden ausgeblendet. Eine Liste der bisher ausgew"ahlten Central Station Kanji wird in der Central Station selbst angezeigt.

\subsection{Codierung visueller Variablen}
Gem"a"s der in der Vorlesung "`Interaktive Informationsvisualisierung"' besprochenen Effektivit"at visueller Variablen, werden in Bushu Mappu verwendet. 

\begin{table}[h]
\begin{tabular}{lllllllll}

Linienl"ange                                & 2D-Position & Ausrichtung & Linienbreite                                                                  & Gr"o"se                                  & Form                           & Kr"ummung & Farbton & Intensit"at \\
keine, nicht charakteristisch f"ur Tubemaps & keine       & keine       & Rahmen gesuchter Kanji: dicker, breite Linienb"undel: mehr gemeinsame Radikale & Central Station: gro"s, Station: normal & Quadrat: Kanji, Kreis: Radikal & keine    & Radikal & keine      \\
\end{tabular}
\end{table}

\subsection{shneiderman mantra}
\subsubsection{Overview}
\subsubsection{Zoom}
\subsubsection{Filter}
\subsubsection{Details-on-demand}
\subsubsection{Relate}
\subsubsection{History}
\subsubsection{Extract}


\section{Herausfoderungen}
Da Haltestellen einen festen Ort haben, der sich nicht "andert, wird dieser als sehr wichtiger Faktor ("`initial embedding"') f"ur das Layouting in Automatic Layout of Metro Maps using Multicriteria Optimisation von Stott (\cite{automaticlayoutmetro08}) beschrieben. Dabei kann sich die Position geringfügig verändern, zum Beispiel werden Entfernungen zwischen Haltestellen auf dem Plan teilweise nicht in dem eigentlich vorhanden Abstand dargestellt, sondern der Abstand wird angeglichen, um ein uniformes Aussehen des Plans zu gewährleisten. Diese Verortung existiert bei Kanji in diesem Mass e nicht und daher müssen Richtungen und Positionen durch einen Algorithmus festgelegt werden. \\
"Ahnlich verh"alt es sich mit der Kreuzung von Linien auss erhalb von Haltestellen. Vor allem bei Schienenverkehr findet eine solche Kreuzung selten statt, meist bedienen Linien stattdessen mindestens eine gemeinsame Haltestelle. Da eine physische Begrenzung dieser Art bei dem verwendeten Datensatz nicht vorhanden ist, m"ussen Linien dort aufwendig per Routingalgorithmus gelegt werden, um "Uberschneidungen – soweit m"oglich – zu vermeiden.

\subsection{L"osungsvorschl"age}
Da der Datensatz mit knapp 2000 Kanji deutlich gr"oss er ist, als die Haltestellen der Dresdner Verkehrsbetriebe(\cite{dvbag})
\paragraph{Clustering}
Bei der Linearen Anordnung werden alle Radikale auf einer x-Achse (wie beim kartesischen Koordinatensystem) angeordnet und "uber ihnen alle Kanji. Diese k"onnen nach Kriterien wie Strichanzahl, Schuljahr oder H"aufigkeit auf der y-Achse sortiert werden. Weiterhin sollten diese grob in Cluster aufgeteilt werden, um sie nahe an den Radikalen zu halten, aus denen sie bestehen beziehungsweise zu denen sie geh"oren. Dies bedeutet, dass die Kanji je nach Menge gleicher gemeinsamer Radikale einem Cluster zugeordnet werden. \\
Hierbei sollte zwischen jedem Kanji ausreichend Platz gelassen werden, um im Zweifelsfall alle 237 Linien zwischen zwei Kanji hindurch zu f"uhren. Ist das Routing zwischen allen Kanji beendet, so k"onnen einige dieser L"ucken wieder verkleinert werden, vorausgesetzt, dass sich dadurch nicht (noch mehr) Linien kreuzen. 
Sehr nachteilig dabei ist der gross e Platzbedarf, der sich durch große ungenutzte Zwischenr"aume ergibt. Des Weiteren hat sich gezeigt, dass Clusterbildung nicht ausreicht um "Uberschneidungen von Linien zu verhindern. \\
Um diese Probleme zu umgehen hat das Team mit einer radialen Anordnung von Radikalen experimentiert, um freie Fl"ache besser zu nutzen, aber dieser Ansatz brachte keine besseren Ergebnisse hervor und wurde deshalb nicht weiter verfolgt.
\paragraph{Reduktion des Datensatzes}
Eine drastische Reduktion des gerade angezeigten Datensatzes wurde dadurch erreicht, dass nur die Radikale eines ausgew"ahlten Kanji gezeigt werden. Auf den Radikal-Linien befinden auss er dem ausgewählten Kanji alle Kanji, in denen das Radikal vorkommt. Durch diese Reduktion lassen sich die (maximal neun) Linien strahlenf"ormig anordnen, wobei Verbindungen zwischen Linien existieren können. Diese werden mit einer Ecke versehen, um möglichst wenige "Uberschneidungen von Linien zu erhalten. 
% rewrite!!



\section{Algorithmen}
% über den optimierer

\section{Frameworks}

warum web?
warum d3, coffeescript?

\end{CJK}

% Literatur, Quellen
\bibliographystyle{abbrv}
\bibliography{bibliography}
% Sollten wir auch ein Glossar einfügen, nur um sicher zu gehen?
\end{document}
