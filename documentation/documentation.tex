\documentclass[color,german]{tudbook}
\usepackage[light,math]{iwona}
\renewcommand*{\dinBold}{\fontencoding{T1}\fontseries{b}\selectfont}
\renewcommand*\thesection{\arabic{section}}
\pagestyle{plain}
\usepackage[utf8]{inputenc}
\usepackage[ngerman]{babel}
\usepackage[hidelinks]{hyperref}
\usepackage{tudthesis, german}
\usepackage[T1]{fontenc}
\usepackage{graphicx}
\usepackage{wrapfig}
\usepackage{hyperref}
\usepackage{float}
\begin{document}

\thesis{Dokumentation}
\author{Paula Sch"oley, Robert Morawa, Gilbert R"ohrbein, Alexandra Wei\ss}
\title{InfoVis}
\supervisedby{Prof. Dr.-Ing. Raimund Dachselt}
\supervisedby{Ulrike Kister, M.Sc., Dipl.-Ing. Ricardo Langner}
\submitdate{8.Juli 2013}
\einrichtung{Fakultät Informatik}
\fachrichtung{}
\institut{Institut für Software- und Multimediatechnik}
\professur{Multimedia-Technologie}
\logofilename{img/imllogo}

\maketitle

\tableofcontents 
\newpage

\section{Einführung}
\subsection{Kanji Schriftsystem}
\subsection{Tube Maps}
\subsection{W-Fragen}
Warum Tube Map für Kanji?
\begin{itemize}
\item klare "Anfangsstation" durch Radikale
\item Exploration ("welche Radikale sollte ich zuerst lernen")
\item Bessere Übersicht über Verlauf als bei einer Darstellung in einem Graphen
\end{itemize}

\section{Algorithmen}
\section{Offene Tickets}


\end{document}
