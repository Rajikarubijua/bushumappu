\subsection{Kanji Schriftsystem}
Die japanische Schrift besteht aus drei Alphabeten, wobei eines aus Logogrammen - sogenannten Kanji - besteht und die anderen beiden Silbenalphabete sind. Kanji stammen aus der chinesischen Sprache und wurden von den Japanern "ubernommen und haben sich seitdem sozusagen parallel weiter entwickelt. In einem Satz werden Kanji Zeichen unter Anderem f"ur den Wortstamm und die Silbenalphabete f"ur das "`Auff"ullen"' (beispielsweise Negation oder Zeitformen) verwendet. Hiragana und Katakana, also die Silbenalphabete, k"onnen aber als Ersatz f"ur ein Kanji verwendet werden, indem sie die Aussprache (die Lesung) des Kanji beschreiben. \\
Kanji bestehen aus einzelnen Elementen, sogenannten Radikalen, die nach gewissen Regeln miteinander kombiniert werden k"onnen. Manche der Radikale sind bereits bedeutungstragend, wie beispielsweise das Radikal \emph{Wasserradikal}, welches f"ur Kontext "`Wasser"' steht. Somit ist es beispielsweise in den W"ortern f"ur "`Flut"' oder auch "`Meer"' enthalten. 
%Bitte Radikal noch einf"ugen

\subsection{Tube Maps}
Tube Maps oder Transit Maps (deutsch: Liniennetzplan) sind eine Form der Visualisierung, die f"ur Linien "offentlichen Nahverkehrs - also Busse, U- und Strass enbahnen - verwendet wird. Bisher werden Tube Maps nur von Menschen gefertigt, da viele verschiedene Faktoren das Aussehen und die Eigenschaften der Map beeinflusen.  
\subsubsection{Eigeschaften}
\begin{itemize}
\item Es existieren \emph{Stationen}, die verschiedene Auspr"agungen besitzen k"onnen (wie "`Endstation"' oder Station, die einen Umstieg zwischen mehreren Linien erm"oglicht). 
\item Stationen liegen auf \emph{Linien}, die diese miteinander verbinden. Linien entsprechen den verlegten Gleisen beziehungsweise Routen, die die einzelnen Stationen verbinden. Es gibt die M"oglichkeit, verschiedene Arten von Linien visuell zu unterscheiden, um zum Beispiel zwischen Strass enbahn- und Buslinie zu unterscheiden.
% Linien k"onnen sich auch teilen?!
\item In \cite{automaticlayoutmetro08} werden als weiterer Bestandteil der Map \emph{Landmarks}, also Orientierungspunkte wie Flugh"afen, Bahnh"ofe, oder sehr wichtige touristische Orte, genannt. Dabei helfen sie bei der Orientierung, indem geographische Bez"uge zwischen Stationen und Landmarks hergestellt werden.
\item \emph{Zonen} stellen in Tube Maps meist Tarifzonen dar, k"onnen aber auch weitere Eigenschaften visualisieren, die sich durch Entfernung codieren lassen. 
\end{itemize}
Tube Maps grenzen sich insofern von Node-Link Diagrammen ab, als dass keine Zyklen entstehen k"onnen und Linien sich selten teilen. Ersteres bedeutet auf eine Tube Map "ubertragen, dass keine Schleifen mit mehreren Stationen existieren (abgesehen von n"otigen Wendeschleifen), sondern die Stationen in ihrer festen Reihenfolge bedient werden. 
% hmm die ringbahn in berlin?! mist... aber wie grenzen sich die tube maps sonst ab?

\subsubsection{Weitere Verwendungszwecke}
\label{tm:verwendungszwecke}
Wie in \cite{automaticlayoutmetro08} beschrieben, gibt es auch weitere Anwendungsgebiete f"ur Tube Maps, wie Projektpl"ane, Karten von Verbindungen zwischen Objekten wie beispielsweise Programmiersprachen, Aufbau von Webseiten, sowie Metabolic Pathways genannt. Eigene Recherche des Teams hat "ahnliche Ergebnisse ergeben, wobei nicht immer bei allen Ergebnissen klar war, warum diese Form der Visualsierung gew"ahlt wurde. 

\section{W-Fragen}
% bitte besseren namen einfallen lassen!!!
\subsection{Warum "uberhaupt Kanji?}
Die japanische Schrift ist streng hierarchisch und modular aufgebaut. Der Datensatz ist begrenzt, da keine neuen Kanji "`erfunden"' werden, und er l"asst sich reduzieren auf sogenannte \emph{Jouyou} Kanji. Jouyou sind Zeichen, die besonders h"aufig im Alltagsleben verwendet werden, und die somit ein Gross teil der Japaner beherrscht. Im Vergleich zur Gesamtzahl aller Kanji beschr"anken sich die Jouyou auf knapp 2000 Zeichen, was eine starke Reduzierung des Datensatzes erm"oglicht.
Diese Reduzierung ist nicht nur von praktischer Bedeutung, es ist auss erdem f"ur Interessierte, die die Sprache lernen wollen, ein guter Startpunkt. 

\subsection{Warum Tube Map f"ur Kanji?}
\begin{itemize}
\item klare "`Anfangsstation"' durch Radikale
\item Exploration ("`welche Radikale sollte ich zuerst lernen?"')
\item Bessere "Ubersicht "uber Verlauf als bei einer Darstellung in einem Graphen
\end{itemize}

\subsection{Wer soll es am Ende verwenden?}
Die Anwendung richtet sich prim"ar an solche, die Kanji lernen m"ochten. Dabei wird davon ausgegangen, dass entweder ein spezielles Kanji gelernt oder nachgeschlagen wird, oder auch ein generelles Interesse an den Zeichen besteht. Soll das Kanji f"ur "`Baum"' gelernt werden, kann man weitere Kanji lernen, die sehr "ahnlich dazu sind. 
\cite{kanjilearningjapanese10} beschreibt eine Lernstrategie, die f"ur Kanji eine besonders gute Anwendung findet. Bei dieser Strategie spielt neben visueller "Ahnlichkeit auch das sogenannte "`grouping"' (also die Bildung von Verkn"upfungen zwischen Vokabeln basierend auf Bedeutung oder Konzepten) eine Rolle. 
Diese Erkenntnisse lassen vermuten, dass Tube Maps sich zum Lernen von Kanji eignen.
