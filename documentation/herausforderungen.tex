Bei normaler Tube Map:
\begin{itemize}
\item Anzahl Stationen relativ überschaubar
\item initial embedding gegeben durch geographische Lage der Stationen
\item Überschneiden von Linien eher selten (nicht in dem Sinne vorhanden bei U-Bahnen)
\end{itemize}

\subsection{Lösungsvorschlänge}
Im Folgenden werden Radikale als \emph{Endstationen} und Kanji als \emph{Stationen} bezeichnet, um den Vergleich zu einer üblichen Tube Map leichter zu machen. Dementsprechend verbindet eine Linie eine Endstation mit allen zugehörigen Stationen.
\subsubsection{Lineare Anordnung von Radikalen}
Bei der Linearen Anordnung werden Endstationen auf einer x-Achse (wie beim kartesischen Koordinatensystem) angeordnet und über ihnen alle Stationen. Diese können nach Kriterien wie Strichanzahl, Schuljahr, JPLT Level oder Häufigkeit auf der y-Achse sortiert werden. Weiterhin sollten diese grob in Cluster aufgeteilt werden, um sie nahe an den Endstationen zu halten, aus denen sie bestehen bzw. zu denen sie gehören. Dies bedeutet, dass die Stationen je nach Menge gleicher gemeinsamer Endstationen einem Cluster zuogeordnet werden.

Hierbei sollte zwischen jeder Station ausreichend Platz gelassen werden, um im Zweifelsfall alle 237 Linien zwischen zwei Stationen hindurch zu führen. Ist das Routing zwischen allen Stationen beendet, so können einige dieser Lücken wieder verkleinert werden, vorausgesetzt, dass sich dadurch nicht (noch mehr) Linien kreuzen. 

\paragraph{Vorteile}
???

\paragraph{Nachteile}
\begin{itemize}
\item sehr großer Platzbedarf
\end{itemize}

\subsection{Radiale Anordnung von Radikalen}

\paragraph{Vorteile}
???

\paragraph{Nachteile}
???