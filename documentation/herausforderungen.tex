Da Haltestellen einen festen Ort haben, der sich nicht "andert, wird dieser als sehr wichtiger Faktor ("`initial embedding"') f"ur das Layouting in "`Automatic Layout of Metro Maps using Multicriteria Optimisation"' von \cite{automaticlayoutmetro08} beschrieben. Dabei kann sich die Position ver"andern, zum Beispiel werden Entfernungen zwischen Haltestellen auf dem Plan nicht geographisch korrekt dargestellt, sondern der Abstand wird angeglichen, um ein uniformes Aussehen des Plans zu gewährleisten. Diese Verortung z.B. nach einer Geographie, liegt bei Kanji nicht vor und daher müssen Richtungen und Positionen durch einen Algorithmus festgelegt werden.

"Ahnlich verh"alt es sich mit der Kreuzung von Linien au"serhalb von Haltestellen. Vor allem bei Schienenverkehr findet eine solche Kreuzung selten statt. Meist bedienen Linien stattdessen mindestens eine gemeinsame Haltestelle. Da eine physische Begrenzung dieser Art bei dem verwendeten Datensatz nicht vorhanden ist, m"ussen Linien dort aufwendig per Routingalgorithmus gelegt werden, um "Uberschneidungen soweit m"oglich zu vermeiden.

Da der Datensatz mit knapp 2000 Kanji deutlich gr"o"ser ist, als die Haltestellen der Dresdner Verkehrsbetriebe(\cite{dvbag}) XXX
