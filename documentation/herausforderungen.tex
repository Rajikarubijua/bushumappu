Bei normaler Tube Map:
\begin{itemize}
\item Anzahl Stationen relativ "uberschaubar
\item initial embedding gegeben durch geographische Lage der Stationen
\item "Uberschneiden von Linien eher selten (nicht in dem Sinne vorhanden bei U-Bahnen)
\end{itemize}

\subsection{L"osungsvorschl"ange}
Im Folgenden werden Radikale als \emph{Endstationen} und Kanji als \emph{Stationen} bezeichnet, um den Vergleich zu einer "ublichen Tube Map leichter zu machen. Dementsprechend verbindet eine Linie eine Endstation mit allen zugeh"origen Stationen.
\subsubsection{Lineare Anordnung von Radikalen}
Bei der Linearen Anordnung werden Endstationen auf einer x-Achse (wie beim kartesischen Koordinatensystem) angeordnet und "uber ihnen alle Stationen. Diese k"onnen nach Kriterien wie Strichanzahl, Schuljahr, JPLT Level oder H"aufigkeit auf der y-Achse sortiert werden. Weiterhin sollten diese grob in Cluster aufgeteilt werden, um sie nahe an den Endstationen zu halten, aus denen sie bestehen bzw. zu denen sie geh"oren. Dies bedeutet, dass die Stationen je nach Menge gleicher gemeinsamer Endstationen einem Cluster zuogeordnet werden.

Hierbei sollte zwischen jeder Station ausreichend Platz gelassen werden, um im Zweifelsfall alle 237 Linien zwischen zwei Stationen hindurch zu f"uhren. Ist das Routing zwischen allen Stationen beendet, so k"onnen einige dieser L"ucken wieder verkleinert werden, vorausgesetzt, dass sich dadurch nicht (noch mehr) Linien kreuzen. 

\paragraph{Vorteile}
???

\paragraph{Nachteile}
\begin{itemize}
\item sehr gross er Platzbedarf
\end{itemize}

\subsection{Radiale Anordnung von Radikalen}

\paragraph{Vorteile}
???

\paragraph{Nachteile}
???