Da Haltestellen einen festen Ort haben, der sich nicht "andert, wird dieser als sehr wichtiger Faktor ("`initial embedding"') f"ur das Layouting in "`Automatic Layout of Metro Maps using Multicriteria Optimisation"' von \cite{automaticlayoutmetro08} beschrieben. Dabei kann sich die Position geringf"ugig ver"andern, zum Beispiel werden Entfernungen zwischen Haltestellen auf dem Plan teilweise nicht in dem eigentlich vorhanden Abstand dargestellt, sondern der Abstand wird angeglichen, um ein uniformes Aussehen des Plans zu gew"ahrleisten. Diese Verortung existiert bei Kanji in diesem Ma"se nicht und daher m"ussen Richtungen und Positionen durch einen Algorithmus festgelegt werden. 

"Ahnlich verh"alt es sich mit der Kreuzung von Linien au"serhalb von Haltestellen. Vor allem bei Schienenverkehr findet eine solche Kreuzung selten statt, meist bedienen Linien stattdessen mindestens eine gemeinsame Haltestelle. Da eine physische Begrenzung dieser Art bei dem verwendeten Datensatz nicht vorhanden ist, m"ussen Linien dort aufwendig per Routingalgorithmus gelegt werden, um "Uberschneidungen – soweit m"oglich – zu vermeiden.

Da der Datensatz mit knapp 2000 Kanji deutlich gr"o"ser ist als beispielsweise die Anzahl der Stra"senbahnhaltestellen der Dresdner Verkehrsbetriebe(259 laut \cite{dvbag}), musste vor der Implementierung ein Konzept entwickelt werden, wie die Daten automatisch in Form einer Tube Map korrekt dargestellt werden k"onnen.
