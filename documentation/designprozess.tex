
\subsection{Auswahl eines passenden Datensatzes}
In der ersten Phase des Designprozesses hat das Team sich mit mehreren Datens"atzen besch"aftigt, die mit Hilfe einer Tube Map visualisiert werden k"onnen. Dabei wurden die Datens"atze mittels Abstimmung auf vier reduziert, woraufhin sich jedes Teammitglied mit einem Thema besch"aftigt hat. Die Ergebnisse wurden im Team vorgestellt und nach Abstimmung und Diskussion das Thema f"ur die Visualisierung gew"ahlt.

\paragraph{Webseiten}
Wie im Abschnitt \ref{tm:verwendungszwecke} beschrieben l"asst sich die Struktur einer Webseite mittels einer Tube Map darstellen.  Diese Darstellung ist vor Allem f"ur gro"se und komplexe Webseiten von Vorteil, da ein Benutzer sich so einfach eine "Ubersicht "uber alle vorhanden Inhalte verschaffen kann. Problematisch dabei ist jedoch, die Eigenschaften der Tube Map deutlich hervorzuheben und nicht einen Graphen als Ergebnis zu erhalten. Durch Verlinkungen von Webseiten l"asst sich das Ergebnis nicht mehr als Tube Map darstellen.

Die Idee des Teammitgliedes war daher, h"aufig besuchte Webseiten und nicht die Struktur einer Webseite darzustellen. Oft besucht ein Benutzer einen Fundus von Seiten im Laufe des Tages. Des Weiteren k"onnten Zonen f"ur die grobe Klassifizierung des Inhalts der Seiten eingef"uhrt werden, wie beispielsweise "`Nachrichten"' oder "`Zeitvertreib"'. Dabei k"onnten Linien auch bestimmten Schlagw"ortern, zum Beispiel "`Technik"', zugeordnet werden, um zusammen mit den Zonen Seiten zu beschreiben, die prim"ar "uber Neuigkeiten im Bereich von Technik und IT berichten. 

\paragraph{Bildeigenschaften}
\label{dp:bildeigenschaften}
Ein weiterer Gedanke verfolgte das Ziel, die Eigenschaften von Bildern wie zum Beispiel Gem"alden darzustellen. Hierbei w"urden die Linien Eigenschaften des Gem"aldes und die Stationen das Gem"alde selbst darstellen. Es w"are eine M"oglichkeit, Muster in dem Gebrauch z.B. von Farbe unter Malern oder Fotographen im Zusammenhang mit ihrer Str"omung, der Zeit oder des Bildthemas zu finden. Dabei w"urde die Tubemap sowohl auf X- und Y-Achse in Zonen aufgeteilt. Die X-Achse w"are in diesem Beispielkonzept eine Zeitachse, die Y-Achse weist die im Werk dominierenden Farben aus. Sie w"urden den typischen Zonen einer Tubemap entsprechen.

Bei diesem Layout w"urden die Achsen je nach Inhalt gestreckt oder gestaucht und die Linien stetig von einer Hauptlinie abzweigen. Die Auswahl an Bildeigenschaften, die man so kodieren k"onnte, ist gro"s. Als Interaktionsm"oglichkeit b"ote sich an, K"unstler oder Skalen zu wechseln sowie diverse Filter oder Neuskalierung zu erm"oglichen. 

\paragraph{Parallele Handlungsstr"ange}
Ein dritter Ansatz besch"aftigte sich mit parallel verlaufenden Handlungsstr"angen wie sie beispielsweise in der Fernsehserie Game of Thrones und dessen Buchvorlage vorkommen. Es w"urde Zuschauern oder Fans der Serie eine M"oglichkeit geben, die bisherige Handlung zu rekapitulieren und bisherige Aufeinandertreffen von Charakteren darzustellen. Einzelne Hauptcharaktere erhalten hierbei eine Tubemaplinie, Kapitel bilden Stationen und Handlungsorte die Zonen der Tubemap. Diese w"urde entlang der X-Achse nach von links nach rechts verlaufen, grob an einem Zeitstrahl orientiert oder aber sich radial in alle Richtungen ausbreiten, wobei das die Vergleichbarkeit erschweren w"urde.

Beschriftungen von Stationen k"onnten in dieser Tubemap eine Wortgruppe als Zusammenfassung des Kapitels liefern. Man k"onnte die Stationen bei Point of View-Kapiteln den PoV auch in der Farbe des jeweiligen Charakters festhalten. Icons k"onnten Ereignisklassen wie Tode oder k"ampferische Auseinandersetzungen symbolisieren. Interaktionen k"onnten hier die Filterung, das Eingrenzen eines Zeitbereiches und "ahnliches beinhalten. 


\subsection{Entwicklung des Programmkonzepts}
Die folgenden Konzepte stellen die Ideen des Teams vor der Implementierung, also in der Konzeptionierungsphase dar und sind deshalb nicht immer exakt so in der Implementierung umgesetzt. Sie spigeln also alle Ideen und Vorstellungen des Teams wider. 
\subsubsection{Der gesamte Kanjiraum}
Der erste Gedanke war, dass man die komplette Zahl Kanji anzeigen k"onnte. Aufgrund der hohen Zahl an Radikalen w"urde jedoch das Einf"arben der einzelnen Linien als wichtiges visuelles Element von Tube-Maps nicht m"oglich. W"urde man jede Radikallinie einf"arben, w"urden sich die Farben kaum voneinander abheben. Eine Idee war, dies mittels Hervorhebung zu l"osen, sodass man die Radikallinie selbst oder alle Linien eines Kanji per Selektion einf"arben kann. Kleine Labels, die bei traditionellen Tubemaps genutzt werden, um die Verkehrslinien innerhalb der Karte zu markieren, werden analog dazu genutzt, um die Radikallinien zu bezeichnen.

Innerhalb der Ansicht navigierte man vor allem mit Zooming und Panning. Eine Minimap, die die Position des Fensters in der Karte angibt, sorgt hierbei f"ur eine grobe "Ubersicht. Ideen wurden entworfen, die Kanji hier nach den in ihnen enthaltenen Radikalen zu clustern und diese Cluster erst mittels semantischem Zoom aufzul"osen, um die "uberbordende Komplexit"at zu verringern und die Analyse nach dem Radikalkriterium zu erleichtern. Mit Filtern hat der Nutzer die M"oglichkeit, die Menge der Kanji einzuschr"anken, hierf"ur eignen sich Attribute wie die unterschiedichen japanischen Lesungen, die Bedeutungen, Strichzahl oder H"aufigkeit des Kanji in japanischen Zeitungsartikeln. Eine Suche sollte Kanji anzeigen, die den ausgew"ahlten Kriterien entsprechen, von dieser Suchliste w"are eine Autonavigation zum gesuchten Kanji m"oglich gewesen. Das Kanji wird f"ur eine kurze Zeit farblich hervorgehoben, w"ahrend alle Kanji, auf die die Suchkriterien zutreffen, mit dickerem Rand gezeichnet werden.

Mittels Hover hat der Nutzer die M"oglichkeit, sich ein gew"ahltes Detail eines Kanji in Form eines Tooltips oder Labels darzustellen. Da ein einfacher Hover bei einer komplexeren Tubemap rasch zu ungew"unschten Details f"uhren kann, wurde es mit einer kurzen Verweilzeit von ca. einer halben Sekunde kombiniert. Um Vergleichbarkeit der Kanji herzustellen, f"uhrten wir eine ausblendbare Detailtabelle ein. Dort wird pro Spalte ein ausgew"ahltes Kanji angezeigt, in den Zeilen die dazugeh"origen Details. Auch hier ist es m"oglich, sich zu den dargestellten Kanjis navigieren zu lassen.


\subsubsection{Die Central Station-Ansicht}
Nach l"angerem "Uberlegen wurde klar, dass die bisherige Struktur des Konzeptes nicht zielf"uhrend ist oder sein wird. Man erh"alt nicht einmal einen groben "Uberblick "uber die Kanji, da die Verkn"upfungen der Kanji durch Radikale viel zu komplex ist, um sie auf einer Tubemap abzubilden. Zudem fehlen unter anderem die f"ur eine Tubemap charakteristischen farbigen Linien. Hinzu kam eine "Uberlegung, wie Kanji selbst gelernt werden: Normalerweise wird ein Kanji zum Lernen vorgegeben. Mittels der Tubemap k"onnte man "`verwandte"' Kanji darstellen, die dieselben Radikale teilen wie das zum Lernen ausgew"ahlte. Auf diese Weise soll man das Kanji besser in seinen Kontext einbetten k"onnen.

In diesem Konzept gibt es einen "`Hauptbahnhof"' (Central Station). Dieses Kanji wird in einer vorherigen "Ubersicht ausgew"ahlt, in der alle Kanji gelistet sind und gefiltert werden k"onnen. Diese Central Station unterscheidet sich signifikant: sie ist gr"o"ser und zeigt alle Details des Kanji in der Station an. Es werden nur Linien zu den Radikalen angezeigt, die in dem Kanji enthalten sind. Da ein Kanji maximal neun Radikale enth"alt, ist es nun m"oglich die Linien mit Farben zu unterscheiden. Auch wird ein Gro"steil der m"oglichen "Uberschneidungen auf diese Weise entfernt. Die Radikallinien f"uhren durch alle Kanji, in denen sie enthalten sind.

Ansonsten bleiben viele Ideen des urspr"unglichen Konzeptes erhalten. Farbige, kleine Radikale kennzeichnen die zugeh"origen Linien. Kanji k"onnen in eine Detailtabelle eingetragen, durchsucht und gefiltert werden. Ebenso bleibt die Autonavigation erhalten. Jedoch entf"allt auf diese Weise das Hervorheben der Linien. Stationenbezeichner erfahren eine kleine "Anderung. In der Central Station existiert nun die M"oglichkeit, Detailkategorien auszuw"ahlen, die dort angezeigt werden. 

Zudem wird eine weitere Interaktionsm"oglichkeit mit den Kanji eingef"uhrt. Man kann nun ein Kanji anw"ahlen, um es zu einer neuen Central Station werden zu lassen. Radikallinien, die auch in der neuen Ansicht vorhanden sind, sollen beim "Ubergang nach M"oglichkeit nur minimal ver"andert werden, um die Orientierung des Nutzers zu unterst"utzen. Neue Radikallinien werden in das Layout eingef"ugt. Nicht mehr ben"otigte Radikallinien werden ausgeblendet. Eine Liste der bisher ausgew"ahlten Central Station Kanji wird in der Central Station selbst angezeigt.

\subsection{Codierung visueller Variablen}
Zu Beginn der Designphase hat das Team die in der Vorlesung "`Interaktive Informationsvisualisierung"' besprochenen visuellen Variablen auf die Tauglichkeit dieser f"ur Tube Maps untersucht. Dabei fiel auf, dass viele der m"oglichen Variablen entweder nicht sinnvoll oder sogar nicht umsetzbar sind. Die visuellen Variablen Position, Ausrichtung, Form und Kr"ummung k"onnen f"ur Codierung von Information verwendet nicht werden, da sie sehr stark von der Gestaltung der Tube Map abh"angen. Wie in \ref{dp:bildeigenschaften} angedeutet, k"onnten X- und Y-Achse zwar f"ur die Codierung der Strichzahl von Kanji verwendet werden, allerdings konnten diese im Layoutalgorithmus nicht zus"atzlich umgesetzt werden. Hinzu kommt, dass zum Beispiel die Position von Stationen bei Tube Maps bereits gegeben und dadurch fest ist und deshalb vom Nutzer nicht anders interpretiert wird.

\begin{table}[h]
\begin{tabular}{ll}

\textbf{Visuelle Variable} & \textbf{Verwendung}                                                          \\
Linienl"ange       & keine, nicht charakteristisch f"ur Tubemaps                                    \\
Linienbreite      & Rahmen gesuchter Kanji: dicker; breite Linienb"undel: mehr gemeinsame Radikale \\
Gr"o"se             & Central Station: gro"s; Station: normal                                       \\
Farbton           & Radikal                                                                        \\
Intensit"at       & Filter                                                                          \\
Verbindung    & Zugeh"origkeit zu Radikalen                                                                       \\
\end{tabular}
\end{table}
