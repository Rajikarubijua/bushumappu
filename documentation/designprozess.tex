\subsection{Auswahl eines passenden Datensatzes}
In der ersten Phase des Designprozesses hat das Team sich mit mehreren Datensätzen beschäftigt, die mit Hilfe einer Tube Map visualisiert werden können. Dabei wurden die Datensätze mittels Abstimmung auf vier reduziert, woraufhin sich jedes Teammitglied mit einem Thema beschäftigt hat. Anschließend wurden diese Ergebnisse dem Team vorgestellt und durch Abstimmung dann das Thema für die Visualisierung gewählt
\paragraph{Webseiten}
Wie im Abschnitt \ref{tm:verwendungszwecke} beschrieben lässt sich die Struktur einer Webseite mittels einer Tube Map darstellen.  Diese Darstellung ist vor Allem für große und komplexe Webseiten von Vorteil, da ein Benutzer sich so einfach eine Übersicht über alle vorhanden Inhalte verschaffen kann. Problematisch dabei ist jedoch, die Eigenschaften der Tube Map deutlich hervorzuheben und nicht einen Graphen als Ergebnis zu erhalten. Durch Verlinkungen zwischen verschiedenen Bereichen von Webseiten wird somit schnell eine der wichtigsten Eigenschaften von Tube Maps verletzt, nämlich die Zyklenfreiheit. \\
Die Idee des Teammitgliedes war daher, die häufig besuchten Webseiten und nicht die Struktur einer Webseite darzustellen. Oft besucht ein Benutzer einen Fundus von Seiten im Laufe des Tages. Diese Routine könnte der Linienführung der Linien entsprechen. Des Weiteren könnten Zonen für die grobe Klassifizierung des Inhalts der Seiten eingeführt werden, wie beispielsweise "`Nachrichten"' oder "`Zeitvertreib"'. Dabei könnten Linien auch bestimmten Schlagwörtern, zum Beispiel "`Technik"', zugeordnet werden, um zusammen mit den Zonen Seiten zu beschreiben, die primär über Neuigkeiten im Bereich von Technik und IT berichten. 

\paragraph{Farben}
\paragraph{Parallele Handlungsstränge}

\subsection{Überlegungen zum Ablauf im Programm}
\subsubsection{Wool Spider}
alle kanji zeigen

\subsubsection{Central Station}
central station ansicht, finales design
wie kamen wir drauf? durch erfahrungen und eigene paper

\subsection{Codierung visueller Variablen}
Gemäß der in der Vorlesung "`Interaktive Informationsvisualisierung"' besprochenen Effektivität visueller Variablen, werden in Bushu Mappu planare Dimensionen (=Position)(Anwendung der Filter und dadurch Abstand zur Central Station), Größe (Central Station), Farbe (Radikallinien) und Form (Rund: Radikal, Quadrat: Kanji) verwendet. 

\subsection{shneiderman mantra}
\subsubsection{Overview}
\subsubsection{Zoom}
\subsubsection{Filter}
\subsubsection{Details-on-demand}
\subsubsection{Relate}
\subsubsection{History}
\subsubsection{Extract}

\subsection{Nicht umgesetzte Konzepte}
\subsubsection{Zonen für Angabe von zB Strichzahl}