\subsection{Auswahl eines passenden Datensatzes}
In der ersten Phase des Designprozesses hat das Team sich mit mehreren Datens"atzen besch"aftigt, die mit Hilfe einer Tube Map visualisiert werden k"onnen. Dabei wurden die Datens"atze mittels Abstimmung auf vier reduziert, woraufhin sich jedes Teammitglied mit einem Thema besch"aftigt hat. Die Ergebnisse wurden im Team vorgestellt und nach Abstimmung und Diskussion das Thema f"ur die Visualisierung gew"ahlt.

\paragraph{Webseiten}
Wie im Abschnitt \ref{tm:verwendungszwecke} beschrieben l"asst sich die Struktur einer Webseite mittels einer Tube Map darstellen.  Diese Darstellung ist vor Allem f"ur gross e und komplexe Webseiten von Vorteil, da ein Benutzer sich so einfach eine "Ubersicht "uber alle vorhanden Inhalte verschaffen kann. Problematisch dabei ist jedoch, die Eigenschaften der Tube Map deutlich hervorzuheben und nicht einen Graphen als Ergebnis zu erhalten. Durch Verlinkungen zwischen verschiedenen Bereichen von Webseiten wird somit schnell eine der wichtigsten Eigenschaften von Tube Maps verletzt, n"amlich die Zyklenfreiheit. \\
Die Idee des Teammitgliedes war daher, die h"aufig besuchten Webseiten und nicht die Struktur einer Webseite darzustellen. Oft besucht ein Benutzer einen Fundus von Seiten im Laufe des Tages. Diese Routine k"onnte der Linienf"uhrung der Linien entsprechen. Des Weiteren k"onnten Zonen f"ur die grobe Klassifizierung des Inhalts der Seiten eingef"uhrt werden, wie beispielsweise "`Nachrichten"' oder "`Zeitvertreib"'. Dabei k"onnten Linien auch bestimmten Schlagw"ortern, zum Beispiel "`Technik"', zugeordnet werden, um zusammen mit den Zonen Seiten zu beschreiben, die prim"ar "uber Neuigkeiten im Bereich von Technik und IT berichten. 

\paragraph{Bildeigenschaften}
Ein weiterer Gedanke verfolgte das Ziel, die Eigenschaften von Bildern wie zum Beispiel Gem"alden darzustellen. Hierbei w"urden die Linien Eigenschaften des Gem"aldes und die Stationen das Gem"alde selbst darstellen. Es w"are eine M"oglichkeit, Muster in dem Gebrauch z.B. von Farbe unter Malern oder Fotographen im Zusammenhang mit ihrer Str"omung, der Zeit oder des Bildthemas zu finden. Dabei w"urde die Tubemap sowohl auf X- und Y-Achse in Zonen aufgeteilt. Die X-Achse w"are in diesem Beispielkonzept eine Zeitachse, die Y-Achse weist die im Werk dominierenden Farben aus; sie w"urden den typischen Zonen einer Tubemap entsprechen. \\
Bei diesem Layout w"urden die Achsen je nach Inhalt gestreckt oder gestaucht und die Linien stetig von einer Hauptlinie abzweigen. Eigenschaften von Gem"alden wie die Farbzusammensetzung k"onnten in Tortendiagrammen dargestellt werden, mehrere Gem"alde mit gleichen Eigenschaften zu einer geclustert. Die Auswahl an Bildeigenschaften, die man so kodieren k"onnte, ist gross . Als Interaktionsm"oglichkeiten b"ote sich an, K"unstler oder Skalen zu wechseln sowie diverse Filter oder Neuskalierung vorzunehmen. 

\paragraph{Parallele Handlungsstr"ange}
Ein dritter Ansatz besch"aftigte sich mit parallel verlaufenden Handlungsstr"angen wie sie beispielsweise in der Fernsehserie Game of Thrones und dessen Buchvorlage vorkommen. Es w"urde Zuschauern oder Fans der Serie eine M"oglichkeit geben, die bisherige Handlung zu rekapitulieren und bisherige Aufeinandertreffen von Charakteren darzustellen. Einzelne Hauptcharaktere erhalten hierbei eine Tubemaplinie, Kapitel bilden Stationen und Handlungsorte die Zonen der Tubemap. Diese w"urde entlang der X-Achse nach von links nach rechts verlaufen, grob an einem Zeitstrahl orientiert oder aber sich radial in alle Richtungen ausbreiten, wobei das die Vergleichbarkeit erschweren w"urde. \\
Stationlabels k"onnten in dieser Tubemap eine Wortgruppe als Zusammenfassung des Kapitels liefern. Man k"onnte die Stationen bei Point of View-Kapiteln den PoV auch in der Farbe des jeweiligen Charakters festhalten. Icons k"onnten Ereignisklassen wie Tode oder k"ampferische Auseinandersetzungen symbolisieren. Interaktionen k"onnten hier die Filterung, das Eingrenzen eines Zeitbereiches und "ahnliches beinhalten. 


\subsection{"Uberlegungen zum Ablauf im Programm}
\subsubsection{Wool Spider}
alle kanji zeigen

\subsubsection{Central Station}
central station ansicht, finales design
wie kamen wir drauf? durch erfahrungen und eigene paper

\subsection{Codierung visueller Variablen}
Gem"ass  der in der Vorlesung "`Interaktive Informationsvisualisierung"' besprochenen Effektivit"at visueller Variablen, werden in Bushu Mappu planare Dimensionen (=Position)(Anwendung der Filter und dadurch Abstand zur Central Station), Gr"oss e (Central Station), Farbe (Radikallinien) und Form (Rund: Radikal, Quadrat: Kanji) verwendet. 

\subsection{shneiderman mantra}
\subsubsection{Overview}
\subsubsection{Zoom}
\subsubsection{Filter}
\subsubsection{Details-on-demand}
\subsubsection{Relate}
\subsubsection{History}
\subsubsection{Extract}

\subsection{Nicht umgesetzte Konzepte}
\subsubsection{Zonen f"ur Angabe von zB Strichzahl}