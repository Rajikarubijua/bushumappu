\subsection{Auswahl eines passenden Datensatzes}
In der ersten Phase des Designprozesses hat das Team sich mit mehreren Datensätzen beschäftigt, die mit Hilfe einer Tube Map visualisiert werden können. Dabei wurden die Datensätze mittels Abstimmung auf vier reduziert, woraufhin sich jedes Teammitglied mit einem Thema beschäftigt hat. Die Ergebnisse wurden im Team vorgestellt und nach Abstimmung und Diskussion das Thema für die Visualisierung gewählt.

\paragraph{Webseiten}
Wie im Abschnitt \ref{tm:verwendungszwecke} beschrieben lässt sich die Struktur einer Webseite mittels einer Tube Map darstellen.  Diese Darstellung ist vor Allem für große und komplexe Webseiten von Vorteil, da ein Benutzer sich so einfach eine Übersicht über alle vorhanden Inhalte verschaffen kann. Problematisch dabei ist jedoch, die Eigenschaften der Tube Map deutlich hervorzuheben und nicht einen Graphen als Ergebnis zu erhalten. Durch Verlinkungen zwischen verschiedenen Bereichen von Webseiten wird somit schnell eine der wichtigsten Eigenschaften von Tube Maps verletzt, nämlich die Zyklenfreiheit. \\
Die Idee des Teammitgliedes war daher, die häufig besuchten Webseiten und nicht die Struktur einer Webseite darzustellen. Oft besucht ein Benutzer einen Fundus von Seiten im Laufe des Tages. Diese Routine könnte der Linienführung der Linien entsprechen. Des Weiteren könnten Zonen für die grobe Klassifizierung des Inhalts der Seiten eingeführt werden, wie beispielsweise "`Nachrichten"' oder "`Zeitvertreib"'. Dabei könnten Linien auch bestimmten Schlagwörtern, zum Beispiel "`Technik"', zugeordnet werden, um zusammen mit den Zonen Seiten zu beschreiben, die primär über Neuigkeiten im Bereich von Technik und IT berichten. 

\paragraph{Bildeigenschaften}
Ein weiterer Gedanke verfolgte das Ziel, die Eigenschaften von Bildern wie zum Beispiel Gemälden darzustellen. Hierbei würden die Linien Eigenschaften des Gemäldes und die Stationen das Gemälde selbst darstellen. Es wäre eine Möglichkeit, Muster in dem Gebrauch z.B. von Farbe unter Malern oder Fotographen im Zusammenhang mit ihrer Strömung, der Zeit oder des Bildthemas zu finden. Dabei würde die Tubemap sowohl auf X- und Y-Achse in Zonen aufgeteilt. Die X-Achse wäre in diesem Beispielkonzept eine Zeitachse, die Y-Achse weist die im Werk dominierenden Farben aus; sie würden den typischen Zonen einer Tubemap entsprechen. \\
Bei diesem Layout würden die Achsen je nach Inhalt gestreckt oder gestaucht und die Linien stetig von einer Hauptlinie abzweigen. Eigenschaften von Gemälden wie die Farbzusammensetzung könnten in Tortendiagrammen dargestellt werden, mehrere Gemälde mit gleichen Eigenschaften zu einer geclustert. Die Auswahl an Bildeigenschaften, die man so kodieren könnte, ist groß. Als Interaktionsmöglichkeiten böte sich an, Künstler oder Skalen zu wechseln sowie diverse Filter oder Neuskalierung vorzunehmen. 

\paragraph{Parallele Handlungsstränge}
Ein dritter Ansatz beschäftigte sich mit parallel verlaufenden Handlungssträngen wie sie beispielsweise in der Fernsehserie Game of Thrones und dessen Buchvorlage vorkommen. Es würde Zuschauern oder Fans der Serie eine Möglichkeit geben, die bisherige Handlung zu rekapitulieren und bisherige Aufeinandertreffen von Charakteren darzustellen. Einzelne Hauptcharaktere erhalten hierbei eine Tubemaplinie, Kapitel bilden Stationen und Handlungsorte die Zonen der Tubemap. Diese würde entlang der X-Achse nach von links nach rechts verlaufen, grob an einem Zeitstrahl orientiert oder aber sich radial in alle Richtungen ausbreiten, wobei das die Vergleichbarkeit erschweren würde. \\
Stationlabels könnten in dieser Tubemap eine Wortgruppe als Zusammenfassung des Kapitels liefern. Man könnte die Stationen bei Point of View-Kapiteln den PoV auch in der Farbe des jeweiligen Charakters festhalten. Icons könnten Ereignisklassen wie Tode oder kämpferische Auseinandersetzungen symbolisieren. Interaktionen könnten hier die Filterung, das Eingrenzen eines Zeitbereiches und ähnliches beinhalten. 


\subsection{Überlegungen zum Ablauf im Programm}
\subsubsection{Wool Spider}
alle kanji zeigen

\subsubsection{Central Station}
central station ansicht, finales design
wie kamen wir drauf? durch erfahrungen und eigene paper

\subsection{Codierung visueller Variablen}
Gemäß der in der Vorlesung "`Interaktive Informationsvisualisierung"' besprochenen Effektivität visueller Variablen, werden in Bushu Mappu planare Dimensionen (=Position)(Anwendung der Filter und dadurch Abstand zur Central Station), Größe (Central Station), Farbe (Radikallinien) und Form (Rund: Radikal, Quadrat: Kanji) verwendet. 

\subsection{shneiderman mantra}
\subsubsection{Overview}
\subsubsection{Zoom}
\subsubsection{Filter}
\subsubsection{Details-on-demand}
\subsubsection{Relate}
\subsubsection{History}
\subsubsection{Extract}

\subsection{Nicht umgesetzte Konzepte}
\subsubsection{Zonen für Angabe von zB Strichzahl}