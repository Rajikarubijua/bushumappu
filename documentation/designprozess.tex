\subsection{Auswahl eines passenden Datensatzes}
In der ersten Phase des Designprozesses hat das Team sich mit mehreren Datens"atzen besch"aftigt, die mit Hilfe einer Tube Map visualisiert werden k"onnen. Dabei wurden die Datens"atze mittels Abstimmung auf vier reduziert, woraufhin sich jedes Teammitglied mit einem Thema besch"aftigt hat. Die Ergebnisse wurden im Team vorgestellt und nach Abstimmung und Diskussion das Thema f"ur die Visualisierung gew"ahlt.

\paragraph{Webseiten}
Wie im Abschnitt \ref{tm:verwendungszwecke} beschrieben lässt sich die Struktur einer Webseite mittels einer Tube Map darstellen.  Diese Darstellung ist vor Allem für große und komplexe Webseiten von Vorteil, da ein Benutzer sich so einfach eine Übersicht über alle vorhanden Inhalte verschaffen kann. Problematisch dabei ist jedoch, die Eigenschaften der Tube Map deutlich hervorzuheben und nicht einen Graphen als Ergebnis zu erhalten. Durch Verlinkungen zwischen verschiedenen Bereichen von Webseiten wird somit schnell eine der wichtigsten Eigenschaften von Tube Maps verletzt, nämlich die Zyklenfreiheit.

Die Idee des Teammitgliedes war daher, die häufig besuchten Webseiten und nicht die Struktur einer Webseite darzustellen. Oft besucht ein Benutzer einen Fundus von Seiten im Laufe des Tages. Diese Routine könnte der Linienführung der Linien entsprechen. Des Weiteren könnten Zonen für die grobe Klassifizierung des Inhalts der Seiten eingeführt werden, wie beispielsweise "`Nachrichten"' oder "`Zeitvertreib"'. Dabei könnten Linien auch bestimmten Schlagwörtern, zum Beispiel "`Technik"', zugeordnet werden, um zusammen mit den Zonen Seiten zu beschreiben, die primär über Neuigkeiten im Bereich von Technik und IT berichten. 

\paragraph{Bildeigenschaften}
Ein weiterer Gedanke verfolgte das Ziel, die Eigenschaften von Bildern wie zum Beispiel Gemälden darzustellen. Hierbei würden die Linien Eigenschaften des Gemäldes und die Stationen das Gemälde selbst darstellen. Es wäre eine Möglichkeit, Muster in dem Gebrauch z.B. von Farbe unter Malern oder Fotographen im Zusammenhang mit ihrer Strömung, der Zeit oder des Bildthemas zu finden. Dabei würde die Tubemap sowohl auf X- und Y-Achse in Zonen aufgeteilt. Die X-Achse wäre in diesem Beispielkonzept eine Zeitachse, die Y-Achse weist die im Werk dominierenden Farben aus; sie würden den typischen Zonen einer Tubemap entsprechen.

Bei diesem Layout würden die Achsen je nach Inhalt gestreckt oder gestaucht und die Linien stetig von einer Hauptlinie abzweigen. Eigenschaften von Gemälden wie die Farbzusammensetzung könnten in Tortendiagrammen dargestellt werden, mehrere Gemälde mit gleichen Eigenschaften zu einer geclustert. Die Auswahl an Bildeigenschaften, die man so kodieren könnte, ist groß. Als Interaktionsmöglichkeiten böte sich an, Künstler oder Skalen zu wechseln sowie diverse Filter oder Neuskalierung vorzunehmen. 

\paragraph{Parallele Handlungsstränge}
Ein dritter Ansatz beschäftigte sich mit parallel verlaufenden Handlungssträngen wie sie beispielsweise in der Fernsehserie Game of Thrones und dessen Buchvorlage vorkommen. Es würde Zuschauern oder Fans der Serie eine Möglichkeit geben, die bisherige Handlung zu rekapitulieren und bisherige Aufeinandertreffen von Charakteren darzustellen. Einzelne Hauptcharaktere erhalten hierbei eine Tubemaplinie, Kapitel bilden Stationen und Handlungsorte die Zonen der Tubemap. Diese würde entlang der X-Achse nach von links nach rechts verlaufen, grob an einem Zeitstrahl orientiert oder aber sich radial in alle Richtungen ausbreiten, wobei das die Vergleichbarkeit erschweren würde.

Stationlabels könnten in dieser Tubemap eine Wortgruppe als Zusammenfassung des Kapitels liefern. Man könnte die Stationen bei Point of View-Kapiteln den PoV auch in der Farbe des jeweiligen Charakters festhalten. Icons könnten Ereignisklassen wie Tode oder kämpferische Auseinandersetzungen symbolisieren. Interaktionen könnten hier die Filterung, das Eingrenzen eines Zeitbereiches und ähnliches beinhalten. 


\subsection{Entwicklung des Programmkonzepts}
\subsubsection{Der gesamte Kanjiraum}
Der erste Gedanke war, dass man die komplette Zahl Kanji anzeigen könnte. Aufgrund der hohen Zahlen an Radikalen würde jedoch ein wichtiges Element der Tubemaps entfallen; die Farbe der einzelnen Linien. Würde man jede Radikallinie einfärben, würden sich die Farben kaum voneinander abheben. Eine Idee war, das mittels Hervorhebung zu lösen, sodass man die Radikallinie selbst oder alle Linien eines Kanji per Selektion einfärben kann. Kleine Labels, die bei traditionellen Tubemaps genutzt werden, die Verkehrslinien innerhalb der Karte zu markieren, werden analog dazu genutzt, um die Radikallinien zu bezeichnen. Zur Unterscheidung von Kanji und Radikalen wird die Form der Station verwendet.

Innerhalb der Ansicht navigierte man vor allem mit Zooming und Panning. Eine Minimap, die die Position des Fensters auf dem Canvas angibt, sorgt hierbei für eine grobe Übersicht. Ideen wurden entworfen, die Kanji hier nach den in ihnen enthaltenen Radikalen zu clustern und diese Cluster erst mittels semantischem Zoom aufzulösen, um die überbordende Komplexität zu verringern und die Analyse nach dem Radikalkriterium zu erleichtern. Mit Filtern hat der Nutzer die Möglichkeit, die Menge der Kanji einzuschränken, hierfür eignen sich Attribute wie die unterschiedichen japanischen Lesungen, die Bedeutungen, Strichzahl oder Häufigkeit des Kanji in japanischen Zeitungsartikeln. Eine Suche sollte Kanji anzeigen, die den ausgewählten Kriterien entsprechen, von dieser Suchliste wäre eine Autonavigation zum gesuchten Kanji möglich gewesen. Das Kanji wird für eine kurze Zeit farblich hervorgehoben, während alle Kanji, auf die die Suchkriterien zutreffen, mit dickerem Rand gezeichnet werden.

Mittels Hover hat der Nutzer die Möglichkeit, sich ein gewähltes Detail eines Kanji in Form eines Tooltips oder Stationslabels darzustellen. Da ein einfacher Hover bei einer komplexeren Tubemap rasch zu ungewünschten Details führen kann, wurde es mit einer kurzen Verweilzeit von ca. einer halben Sekunde kombiniert, sodass eine klare Eingabe des Nutzers vorliegen muss. Um weitere Vergleichbarkeit der Kanji herzustellen führten wir eine ausblendbare Detailtabelle ein. Dort wird pro Spalte ein ausgewähltes Kanji angezeigt, in den Zeilen die dazugehörigen Details. Auch hier sollte man automatisch zu den angezeigten Kanji navigiert werden. 


\subsubsection{Die Central Station-Ansicht}
Nach längerem Überlegen merkten wir, dass die bisherige Struktur des Konzeptes nicht zielführend ist oder sein wird. Man erhält nicht einmal einen groben Überblick über die Kanji, da die Struktur der Tubemap zu groß und komplex ist und man beim Hereinzoomen keinerlei Überblick über die restlichen Kanji erhält. Zudem fehlen unter anderem die für eine Tubemap charakteristischen farbigen Linien. Hinzu kam eine Überlegung, wie Kanji selbst gelernt werden. Normalerweise wird ein Kanji zum Lernen vorgegeben; mittels der Tubemap könnte man "`verwandte"' Kanji, die dieselben Radikale teilen wie das zum Lernen ausgewählte Kanji. Auf diese Weise soll man das Kanji besser in seinen Kontext einbetten können.

In diesem Konzept gibt es einen "`Hauptbahnhof"' (Central Station). Dieses Kanji wird in einer vorherigen Übersicht ausgewählt, in der alle Kanji gelistet sind und gefiltert werden können. Diese Central Station unterscheidet sich signifikant; sie ist größer und zeigt alle Details des Kanji in der Station an. Es werden nur Linien zu den Radikalen angezeigt, die in dem Kanji enthalten sind. Da ein Kanji maximal neun Radikale enthält, ist es nun möglich die Linien mit Farben zu unterscheiden. Auch wird ein Großteil der möglichen Überschneidungen auf diese Weise entfernt. Die Radikallinien führen durch alle Kanji, in denen sie enthalten sind.

Ansonsten bleiben viele Ideen des ursprünglichen Konzeptes erhalten. Farbige, kleine Radikale kennzeichnen die zugehörigen Linien. Kanji können in eine Detailtabelle eingetragen, durchsucht und gefiltert werden, ebenso die Autonavigation. Jedoch entfällt auf diese Weise das Hervorheben der Linien. Stationenbezeichner erfahren eine kleine Änderung; in der Central Station existiert nun die Möglichkeit, Detailkategorien auszuwählen, die dort angezeigt werden. Zudem wird eine weitere Interaktionsmöglichkeit mit den Kanji eingeführt. Man kann nun ein Kanji anwählen, um es zu einer neuen Central Station werden zu lassen. Dabei sollen die Radikallinien, die bereits layoutet werden, weil sie in beiden Kanji enthalten sind, nur minimal verändert werden, während die neu hinzugekommenen Radikallinien ins Layout eingefügt werden. Nicht mehr benötigte Radikallinien werden ausgeblendet.

\subsection{Codierung visueller Variablen}
Gem"ass  der in der Vorlesung "`Interaktive Informationsvisualisierung"' besprochenen Effektivit"at visueller Variablen, werden in Bushu Mappu planare Dimensionen (=Position)(Anwendung der Filter und dadurch Abstand zur Central Station), Gr"oss e (Central Station), Farbe (Radikallinien) und Form (Rund: Radikal, Quadrat: Kanji) verwendet. 

\subsection{shneiderman mantra}
\subsubsection{Overview}
\subsubsection{Zoom}
\subsubsection{Filter}
\subsubsection{Details-on-demand}
\subsubsection{Relate}
\subsubsection{History}
\subsubsection{Extract}