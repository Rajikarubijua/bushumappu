\subsection{Kanji Schriftsystem}
Die japanische Schrift besteht aus drei Alphabeten, wobei eines aus Logogrammen - sogenannten Kanji - besteht und die anderen beiden Silbenalphabete sind. Kanji stammen aus der chinesischen Sprache und wurden von den Japanern "ubernommen und haben sich seitdem parallel entwickelt. In einem Satz werden Kanji unter anderem f"ur den Wortstamm und die Silbenalphabete f"ur grammatikalische Beugung (beispielsweise Negation oder Zeitformen) verwendet. Hiragana und Katakana, also die Silbenalphabete, k"onnen aber auch als Ersatz f"ur ein Kanji verwendet werden, indem sie die Aussprache (die Lesung) des Kanji beschreiben. \\
Kanji bestehen aus einzelnen Elementen, sogenannten Radikalen, die miteinander kombiniert werden k"onnen. Wenige der Radikale sind bereits bedeutungstragend, wie beispielsweise das Radikal \emph{氵}, welches f"ur Kontext "`Wasser"' steht. Somit ist es beispielsweise in den W"ortern f"ur "`Flut"'(沔) oder auch "`Meer"'(海) enthalten. 

\subsection{Tube Maps}
Tube Maps oder Transit Maps (deutsch: Liniennetzplan) sind eine Form der Visualisierung, die f"ur Linien "offentlichen Nahverkehrs - also Busse, U- und Stra"senbahnen - verwendet wird. Bisher werden Tube Maps nur von Menschen gefertigt, da viele verschiedene Faktoren das Aussehen und die Eigenschaften der Tube Map beeinflusen.  
\subsubsection{Eigeschaften}
\begin{itemize}
\item Es existieren \emph{Stationen}, die verschiedene Auspr"agungen, wie "`Endstation"' oder "`Umstiegsm"oglichkeiten"' besitzen. 

\item Stationen liegen auf \emph{Linien}, die diese miteinander gemeinsam haben. Linien entsprechen den verlegten Gleisen beziehungsweise Routen, die die einzelnen Stationen verbinden. Es gibt die M"oglichkeit, verschiedene Arten von Linien visuell zu unterscheiden, um zum Beispiel zwischen Stra\ss enbahn- und Buslinie zu unterscheiden. Des Weiteren d"urfen die Linien nur horizontal, vertikal oder im 45$^{\circ}$ Winkel zueinander sein. 

Linien k"onnen sich ab einer Station aufteilen, um zu zwei alternativen Endstationen zu fahren. 

Hinzu kommen Ringbahnen, bei denen die Endstation der Anfangsstation entspricht.

\item In \cite{automaticlayoutmetro08} werden als weiterer Bestandteil der Map \emph{topographische Metadaten}, also Orientierungspunkte wie Fl"usse, Flugh"afen, Bahnh"ofe, oder sehr wichtige touristische Orte genannt. Dabei helfen sie bei der Orientierung, indem geographische Bez"uge zwischen Stationen und Landmarks hergestellt werden.

\item Die Beschriftung der vorhandenen Stationen und Linien eines Liniennetzes spielt eine wichtige Rolle beim Kennenlernen und Verinnerlichen der vorhandenen Linien. H"aufig wird eine Farbcodierung f"ur die Beschriftung der Linien verwendet, w"ahrend dies bei Stationen nicht der Fall ist und diese direkt mit Text versehen werden. 

\item \emph{Zonen} stellen in Tube Maps meist Tarifzonen dar, k"onnen aber auch weitere Eigenschaften visualisieren, die sich durch Entfernung kodieren lassen. 
\end{itemize}

\subsubsection{Weitere Verwendungszwecke}
\label{tm:verwendungszwecke}
Wie in \cite{automaticlayoutmetro08} beschrieben, gibt es auch weitere Anwendungsgebiete f"ur Tube Maps, wie Projektpl"ane, Karten von thematischen Verbindungen zwischen B"uchern \cite{oreilly}, Aufbau von Webseiten, sowie Metabolic Pathways. Eigene Recherche des Teams hat "ahnliche Ergebnisse ergeben, wobei nicht immer bei allen Visualisierungen klar war, warum eine Visualisierung in Form einer Tube Map gew"ahlt wurde oder sich die Daten besoders dafür eignen.

\section{Motivation}
\subsection{Warum Kanji?}
Die japanische Schrift ist streng hierarchisch und modular aufgebaut. Der Datensatz ist begrenzt, da keine neuen Kanji "`erfunden"' werden, und er l"asst sich reduzieren auf sogenannte \emph{Jouyou} Kanji. Jouyou sind Zeichen, die besonders h"aufig im Alltagsleben verwendet werden, und die somit ein Gro\ss teil der Japaner beherrscht. Im Vergleich zur Gesamtzahl aller Kanji (das Team verwendet eine Datei mit 12000 Kanji) beschr"anken sich die Jouyou auf knapp 2000 Zeichen, was eine starke Reduzierung des Datensatzes erm"oglicht.
Diese Reduzierung ist nicht nur von praktischer Bedeutung, es ist au\ss erdem f"ur Interessierte, die die Sprache lernen wollen, ein guter Startpunkt. 

\subsection{Warum eine Tube Map f"ur Kanji?}
Durch den modularen Aufbau von Kanji k"onnen Radikale als Endstationen interpretiert werden, deren Linien durch mehrere Stationen (also Kanji) verlaufen. Dadurch ergibt sich der Aufbau eines Kanji anhand der Linien, die die Station bedienen. 

Lernende k"onnen somit existierendes Wissen "uber Radikale und Kanji zur Orientierung verwenden, oder aber neue Zeichen kennenlernen, indem sie der Linie eines bekannten Radikals folgen. Anhand der H"aufigkeitsstatistik, die f"ur die Alltags-Kanji vorhanden ist, kann ein Benutzer sich anhand h"aufiger Kanji darin enthaltene Radikale merken und diese verwenden, um weitere Kanji zu lernen. 
\paragraph{Wer soll es am Ende verwenden?}
Die Anwendung richtet sich prim"ar an solche, die Kanji lernen m"ochten. Dabei wird davon ausgegangen, dass entweder ein spezielles Kanji gelernt oder nachgeschlagen wird, oder auch ein generelles Interesse an den Zeichen besteht. Soll beispielsweise das Kanji f"ur "`Baum"' gelernt werden, kann man weitere Kanji lernen, die sehr "ahnlich dazu sind.  
\cite{kanjilearningjapanese10} beschreibt eine Lernstrategie, die f"ur Kanji eine besonders gute Anwendung findet. Bei dieser Strategie spielt neben visueller "Ahnlichkeit auch das sogenannte "`grouping"' (also die Bildung von Verkn"upfungen zwischen Vokabeln basierend auf Bedeutung oder Konzepten) eine Rolle. Diese Erkenntnisse lassen vermuten, dass Tube Maps sich zum Lernen von Kanji eignen.
