\section{Implementierung}

Die Software ist eine vollst"andig Client-seitige Webanwendung. Diese
Entscheidung wurde in Abw"agung gegen"uber C++ mit Qt 5 getroffen.
Ausschlaggebende Argumente waren die Verf"ugbarkeit von CSS und SVG zur
Darstellung im Gegensatz zu imperativen und objektorientierten Ans"atzen
sowie das einfache Zeigen und Teilen mit anderen Menschen "uber einen
Link und einen Browser.

\subsection{Entwicklungsumgebung}

Mit der Entscheidung zur Webanwendung war die erste Frage, welche
Programmiersprache die vorherrschende sein sollte. Zwischen JavaScript,
CoffeeScript, Java und Haxe fiel die Wahl auf
\href{http://coffeescript.org/}{CoffeeScript 1.6.2}. Stark typisierte
Sprachen, die zu JavaScript kompilieren wie Java und Haxe, haben das
Problem, dass JavaScript Bibliotheken wie d3.js, einen abschreckenden
Aufwand erfordern, die Interfaces zu definieren, so dass der Vorteil der
starken Typisierung auch genutzt werden kann. CoffeeScript ist
JavaScript in sofern "uberlegen, da es ein Klassen-Pattern bereits in die
Sprache integriert hat und auch vor anderen Unklarheiten in JavaScript
sch"utzt.

Zur Darstellung selbst wird \href{http://www.w3.org/TR/SVG11/}{SVG 1.1}
und \href{http://www.w3.org/TR/css-2010/}{CSS 2010} verwendet, um mit
Hilfe von CSS schnell visuell ansprechende Effekte und Transitionen nach Bedarf zu
definieren und sie vom Browser bereits effizient umgesetzt werden. CSS
Effekte sind jedoch nur auf DOM Elemente anwendbar, was wiederum SVG
als Darstellungsframework verlangt. \href{http://d3js.org/}{d3.js 3.1.6}
ist darauf ausgelegt, DOM-Elemente basierend auf Daten zu erstellen.
Andere Darstellungsbibliotheken f"ur Webanwendungen legen entweder ihren
Fokus nicht auf SVG und CSS oder sind nicht datengebunden.

Zwischen Firefox 23 und \href{http://www.chromium.org/Home}{Chromium 28}
fiel die Wahl auf letzteren, da der Chromium Browser SVG mit Text effizient
darstellt. Der Firefox st"o"st hier an eine Grenze.

Weitere verwendete Technologien sind
\href{http://requirejs.org/}{require.js 2.1.6} um die Software in Module
zu teilen, \href{https://github.com/fynyky/reactor.js}{reactor.js commit
cdbf994} als marginale Abh"angigkeit, die lediglich f"ur einen simplen
Fall ausprobiert wurde,
\href{https://www.gnu.org/software/make/}{GNU Make 3.81} um typische
Entwicklungsprozesse zu automatisieren und
\href{http://python.org/}{Python 3.2.3} als lokaler Webserver.

\subsection{Tubemap Layout}

Die Software f"ur ein ansehnliches Layout f"ur die Tubemap zu
implementieren, gestaltete sich als sehr schwierig. Die Doktorarbeit von
\cite{automaticlayoutmetro08} nahmen wir als Inspiration, jedoch eignete
sich diese Arbeit lediglich f"ur einen groben Ansatz. Im Detail mussten
viele andere Entscheidungen getroffen werden und andere Ans"atze
ausprobiert werden.

Prinzipiell besteht der Layout Prozess aus zwei Phasen. Zuerst wird eine
initiale Einbettung bestimmt. Dabei sollen Positionen der Knoten
bestimmt werden und wie diese untereinander verbunden sind. Danach wird
der entstandene Graph optimiert. Dies geschieht durch eine
kontinuierlich Bewertung und Ver"anderung um eine bessere Bewertung zu
erzielen. Die Bewertung richtet sich nach mehreren Kriterien und Regeln.

Konkret ist das Ergebnis der initialen Einbettung, dass das zentrale
Kanji der Mittelpunkt ist. Die Radikale des zentralen Kanji werden hier als
relevante Radikale bezeichnet. Von dem zentralen Kanji ausgehend, denkt
man sich f"ur jedes relvante Kanji Strahlen in 90° und 45° Winkel
zueinander. F"ur jedes relevante Radikal werden die Kanji genommen, die
dieses Radikal enthalten und eingeteilt in solche, die von allen
relevanten Radikalen nur eins enthalten (Lo-Kanji), und solche, die
mehrere relevante Radikale enthalten (Hi-Kanji). Hi-Kanji werden unter
den Strahlen verteilt, Lo-Kanji am entprechenden Strahl hinten
angef"ugt. Alle Kanji haben nun eine Position. Um Kanji miteinandner
zu verbinden, werden f"ur jedes relevante
Radikal Verbindungen nacheinander gezogen. Vom zentralen Kanji aus werden alle Hi-Kanji des zum Radikal
zugeh"origen Strahls verbunden und anschlie"send wird jeder Strahl
durchgegangen and alle dortigen Hi-Kanji, die das aktuelle Radikal
enthalten, nacheinander verbunden. Zum Schluss wird noch eine Verbindung
zu den dazugeh"origen Lo-Kanji gezogen.

Die initiale Einbettung ist ausgesprochen wichtig f"ur eine erfolgreiche
Optimierung. Die Optimierung nach mehreren Kriterien wird von
\cite{automaticmetromap11} beschrieben. Ein Teil dieser Optimierung ist
in der vorliegenden Software implementiert, jedoch in entscheidenden
Aspekten abge"andert um schneller zu einem guten Ergebnis zu kommen, die
Entwicklung zu vereinfachen und auf spezielle Bed"urfnisse der
Kanji-Tubemap einzugehen. Die Optimierung findet parallel in einem
\href{http://www.whatwg.org/specthes/web-apps/current-work/multipage/workers.html}{Worker}
statt um die Interaktivit"at der Anwendung nicht zu behindern.
Kontinuierlich wird ein Knoten im Graph ausgew"ahlt und gepr"uft ob in der
Umgebung eine Position f"ur den Knoten ist, bei der die Bewertung des
Graphen besser wird. In die Bewertung flie"sen Regeln und
Qualit"atskriterien ein. Weniger Regeln zu verletzen ist stets besser als
eine h"ohere Qualit"at zu erreichen.

Folgende Regeln sollen m"oglichst nicht verletzt werden:

\begin{itemize}
\item
  keine Kanten, d"urfen unter einem Knoten hindurchf"uhren, die nicht zu dem
  Knoten selbst geh"oren (\texttt{wrongEdgesUnderneath})
\item
  der Knoten darf dem Knoten f"ur das zentrale Kanji nicht zu nahe kommen
  (\texttt{tooNearCentralNode})
\item
  keine Kanten, die andere Kanten kreuzen (\texttt{edgeCrossings})
\end{itemize}

Folgende Qualit"atskriterien flossen in die Version der Software ein, die
zur finalen Pr"asentation gezeigt wurde:

\begin{itemize}
\item
  aneinander liegende Kanten sollen m"oglichst gerade sein
  (\texttt{lineStraightness})
\item
  Kanten sollen m"oglichst kurz sein (\texttt{lengthOfEdges})
\end{itemize}
